%%----------------------------------------------------------------------------80
%% Section title
%%----------------------------------------------------------------------------80
\section{Introducción}


%%----------------------------------------------------------------------------80
%% Section Content
%%----------------------------------------------------------------------------80
Fortran es un lenguaje de programación desarrollado en los años 50 y activamente utilizado desde entonces, principalmente, en aplicaciones científicas y análisis numérico. Ha sido ampliamente adoptado por la comunidad científica para escribir aplicaciones con cómputos intensivos de alto rendimiento.

Desde 1958 ha pasado por varias versiones, entre las que destacan FORTRAN II, FORTRAN IV, FORTRAN 77, Fortran 90, Fortran 95, Fortran 2003 y Fortran 2008. Si bien el lenguaje era inicialmente un lenguaje imperativo, las últimas versiones incluyen elementos de la programación orientada a objetos.

El curso de Fortran (nivel intermedio), está orientado a que se afiancen los conocimientos básicos en Fortran, y su uso con en la Programación Orientada a Objetos. Así brinda una introducción a los tópicos especializados de Interoperabilidad y Computación de Alto Rendimiento. Para este curso se emplearan características bien conocidas de pero poco empleadas de Fortran 90/95 y se profundizará en las características Fortran 2003/2008.
