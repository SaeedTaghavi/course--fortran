%-----------------------------------------------------------------------------80
% CONTENT
%-----------------------------------------------------------------------------80
%Definición-------------------------------------------------------------------80

\subsection{Variables y constantes}

\begin{frame}[fragile]{Variables y constantes}
\onslide<0->\textbf{Declaración de variables}
 \begin{itemize}[<+(0)->]
  \item La declaración de una o más variables del mismo tipo está dada por la sintaxis\\ 
   \centering <tipo> ,[<atributo(s)>] [::] <variable(s)> [=<valor>]
  \item Algunos atributos son:\\ 
  parameter, save, pointer, target, allocatable, dimension, public, private, external, intrinsic, optional.
  \vspace{6pt}
  \item []
   \begin{minted}[linenos,autogobble]{fortran}
    CHARACTER(len= 4), PARAMETER :: prompt = ">>> "
    CHARACTER(len= *), PARAMETER :: message = "Ingresa tu primer nombre [máx 20 car]:"
   \end{minted}
  \item[] \rightline {\textit{Véase strings.f95}}
 \end{itemize}
\end{frame}

\begin{frame}[fragile]{Variables y constantes}
\onslide<0->\textbf{Declaración de constantes}
 \begin{itemize}[<+(0)->]
  \item Si se requiere que una variable que tome un valor definido no suceptible de cambio, se utiliza el atributo parameter. \\ 
   \begin{minted}[linenos,autogobble]{fortran}
    CHARACTER, PARAMETER :: NewLine = CHAR(10)
   \end{minted}
   \item[] \rightline {\textit{Véase strings.f95}}
   \vspace{-3pt}
   \item Las variables pueden ser definidas en función de constantes mediante el atributo parameter.
 \end{itemize}
\end{frame}
