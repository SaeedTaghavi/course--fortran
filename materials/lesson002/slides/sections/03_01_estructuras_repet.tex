%-----------------------------------------------------------------------------80
% CONTENT
%-----------------------------------------------------------------------------80
%Definición-------------------------------------------------------------------80

\subsection{Repeticiones: DO, DO WHILE, DO FROM X TO Y}


\begin{frame}[fragile]{Repeticiones: DO, DO WHILE, DO}  
 \begin{itemize}[<+(0)->]
  \item Aqupi DO 
  
%  \item Una función intrínseca monoargumental se expre
 %     \centering <función> (<argumento>)\\ 
  %\item Las funciones intrínsecas pueden no tener argumento o tener varios.
  %\item Las funciones biargumentales más destacadas son la función CMPLX y la función MOD.
  %\item [-] CMPLX asigna un valor complejo a partir de valores reales. 
   %\begin{minted}[linenos,autogobble]{fortran}
    %!Sea z = (x,y) una variable compleja
 %   CMPLX (<expres_1>, <expres_2>) !expres son de tipo REAL
 %  \end{minted}
%  \vspace{0.2cm}
%  \item [-] MOD es el valor que corresponde al resto de una división, es decir\\
%      \centering $n = n/m + MOD(n, m) $
%   \begin{minted}[linenos,autogobble]{fortran}
%    !Sea n de tipo INTEGER o REAL y m del mismo tipo y distinto de 0
%    MOD (<expres_1>, <expres_2>) !expres son de tipo INTEGER o REAL
%   \end{minted}
  
 \end{itemize}
\end{frame}



