%-----------------------------------------------------------------------------80
% CONTENT
%-----------------------------------------------------------------------------80
%Definición-------------------------------------------------------------------80

\subsection{Operaciones elementales}


\begin{frame}[fragile]{Operaciones de comparación}  
 \begin{itemize}[<+(0)->]
  \item La operación de comparación se expresa de la forma \\
      \centering <expresión\_1> <operador> <expresión\_2>
  \item Para variables de tipo COMPLEX solo son válidos los operadores == y /= .
  \item Los operadores de comparación son variables de tipo LOGICAL.
    \vspace{0.4cm}
  \item []
    \begin{table}[]
    \centering
    \label{Tabla_comparacion}
    \begin{tabular}{|l|l|l|l|}
    \hline
    Fortran 90  & Fortran 77    & Significado                  \\ \hline
    ==          & .eq.          & es igual a                   \\ \hline
    /=          & .ne.          & no es igual a                \\ \hline
    >           & .gt.          & es estrictamente mayor a     \\ \hline
    >=          & .ge.          & es mayor o igual a           \\ \hline
    <           & .lt.          & es estrictamente menor a     \\ \hline
    <=          & .le.          & es menor o igual a           \\ \hline              
    \end{tabular}
    \caption*{Operadores de comparación en Fortran}
    \end{table}
 \end{itemize}
\end{frame}


\begin{frame}[fragile]{Operaciones lógicas}  
 \begin{itemize}[<+(0)->]
  \item La operación lógica se expresa de la forma \\
      \centering <expresión\_1> <operador> <expresión\_2>
  \item Las operaciones lógicas se evaluan luego de las operaciones de comparación, de izquierda a derecha.
     \vspace{0.4cm}
  \item []
    \begin{table}[]
    \centering
    \label{Tabla_logica}
    \begin{tabular}{|l|l|l|}
    \hline
    Operador    & Significado     \\ \hline
    .not.       & No              \\ \hline
    .and.       & y               \\ \hline
    .or.        & o               \\ \hline
    .eqv.       & equivalente     \\ \hline
    .neqv.      & no equivalente  \\ \hline              
    \end{tabular}
    \caption*{Operadores lógicos en Fortran}
    \end{table}
 \end{itemize}
\end{frame}