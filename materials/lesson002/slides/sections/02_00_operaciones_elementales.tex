%-----------------------------------------------------------------------------80
% SECTION TITLE|
%-----------------------------------------------------------------------------80

\section{Operaciones elementales}  

%-----------------------------------------------------------------------------80
% CONTENT
%-----------------------------------------------------------------------------80
%Definición-------------------------------------------------------------------80

\subsection{Operaciones elementales}

\begin{frame}[fragile]{Operaciones elementales} 
\textbf{Reglas generales}
 \begin{itemize}[<+(1)->]
  \item Las operaciones siguen el orden tradicional de las operaciones matemáticas, es decir, primero los términos entre paréntesis, exponentes, multiplicación y adición.
  \item El uso del símbolo = en el lenguaje Fortran tiene el sentido de asignación mientras que en el uso matemático tiene sentido de igualdad. 
  \item La asignación de una variable tiene la sintaxis \\
      \centering <variable> = <expresión>\\
 \end{itemize}
\end{frame}

\begin{frame}[fragile]{Operaciones aritméticas}
 \begin{itemize}[<+(0)->]
  \item Estan permitidas las operaciones entre valores tipo INTEGER, REAL y COMPLEX.
  \item Las operaciones están dadas por adición (+), sustracción (-), multiplicación (*), división (/) y potenciación (**).
  \item Los operandos del mismo tipo y clase resultan en otro del mismo tipo y clase.
 \end{itemize}
\end{frame}

\begin{frame}[fragile]{Operaciones aritméticas}
\onslide<0->\textbf{Operaciones de tipo INTEGER}
 \begin{itemize}[<+(1)->]
  \item Las operaciones de tipo INTEGER manejan números enteros dentro de un rango en $\mathbb{Z}$.
  \item La división se obtiene con un resto; es decir\\
  \centering $\frac{x}{y} = z \Longleftrightarrow  |x| = |z| \cdot |y| + resto $
  \item La potenciación depende del tipo de variable del exponente. 
  $$
  x**n = \left\{ \begin{array}{lcc}
             \underbrace{x \ast x \ast \cdots \ast x}_\text{$n$ veces} &   si  & n > 0 \\
             \\ \frac{1}{x \ast \ast (-n)} &  si & n < 0 \\
             \\ 1 &  si  & x = 0 
             \end{array}
   \right.
  $$
 \end{itemize}
\end{frame}

\begin{frame}[fragile]{Operaciones aritméticas}
\onslide<0->\textbf{Conversión de tipos}
 \begin{itemize}[<+(1)->]
  \item Los enteros son convertidos en reales o complejos.
  \item Los reales son convertidos en complejos.
  \item Los reales o complejos son convertidos en la clase (kind) más alta.
  \item Al asignar valores (=), la parte derecha se evalúa en el tipo y clase correspondiente, luego es convertida al del tipo y clase de la variable al lado izquierdo.
  \vspace{0.2cm}
  \item[]
  \begin{minted}[linenos,autogobble]{fortran}
   INTEGER          :: n, m 
   REAL             :: a, b
   REAL(kind=8)     :: x, y
   COMPLEX          :: c
   COMPLEX(kind=8)  :: z
   :
   a = (x*(n**c))/z
   :
   \end{minted}
 \end{itemize}
\end{frame}


\begin{frame}[fragile]{Operaciones aritméticas}
\onslide<0->\textbf{Conversiones de tipo más significativas}
    \begin{table}[]
    \centering
    \label{Tabla_comparacion}
    \begin{tabular}{|l|l|l|l|}
    \hline
    Conversión      & Mecanismo de Conversión               \\ \hline
    x = n           & x = n                                 \\ \hline
    x = a           & x = a                                 \\ \hline
    n = x           & $  n = \left\{ \begin{array}{lcccc}
                 m &   si  & m \leq x \leq m+1   & y & x \geq 0 \\
             \\ -m &   si  & m \leq -x \leq m+1  & y & x < 0 
             \end{array}
   \right.$                                                 \\ \hline
    a = x           & a = round(x)                          \\ \hline
    a = c           & a = $\mathbb{Re}$(z)                   \\ \hline
    z = x           & z = (x, 0)                            \\ \hline              
    \end{tabular}
    
    \end{table}
\end{frame}