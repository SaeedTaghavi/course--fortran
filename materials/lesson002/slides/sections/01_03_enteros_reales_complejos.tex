%-----------------------------------------------------------------------------80
% CONTENT
%-----------------------------------------------------------------------------80
%Definición-------------------------------------------------------------------80

\subsection{Enteros, reales y complejos}

\begin{frame}[fragile]{Enteros}
\onslide<0->\textbf{Tipos de enteros}
 \begin{itemize}[<+(1)->]
  \item La representación de valores enteros se declara con INTEGER.
  \item Los valores pueden ser guardados usualmente con presición simple, doble o cuádruple.  
  \vspace{0.2cm}
  \item []
   \begin{minted}[linenos,autogobble]{fortran} 
    INTEGER, PARAMETER :: K02 = SELECTED_INT_KIND(2)
    INTEGER, PARAMETER :: K04 = SELECTED_INT_KIND(4)
    INTEGER, PARAMETER :: K08 = SELECTED_INT_KIND(8)
    INTEGER, PARAMETER :: K16 = SELECTED_INT_KIND(16)

    INTEGER(kind=K02) :: I02
    INTEGER(kind=K04) :: I04
    INTEGER(kind=K08) :: I08
    INTEGER(kind=K16) :: I16
   \end{minted}
  \rightline {\textit{Véase kind\_integers.f95}}
 \end{itemize}
\end{frame}

\begin{frame}[fragile]{Reales}
\onslide<0->\textbf{Tipos de reales}
 \begin{itemize}[<+(1)->]
  \item La representación de número reales se declara con REAL y puede ser de presición estándar o simple presición (sp) y de presición superior, doble (dp) o cuádruple (qp) presición en adelante.
  \item La sintaxis para el tipo real es\\ 
   \centering REAL (kind=<np>)
  \vspace{0.2cm}
  \item []
   \begin{minted}[linenos,autogobble]{fortran}
    REAL(kind=p04) :: X04
    REAL(kind=p08) :: X08
    REAL(kind=p16) :: X16
    REAL(kind=p32) :: X32
   \end{minted}
 \rightline {\textit{Véase kind\_real.f95}}
 \end{itemize}
\end{frame}


\begin{frame}[fragile]{Reales}
 \begin{itemize}[<+(0)->]
  \item La notación científica para los reales viene dada por los identificadores "e" (sp), "d" (dp) y "q" (qp).  
  \vspace{0.2cm}
  \item []
   \begin{minted}[linenos,autogobble]{fortran}
    REAL(kind=4)  :: x = 2.e0       !simple presición
    REAL(kind=8)  :: y = 4.d-6      !doble presición
    REAL(kind=16) :: z = -8.q-1000  !cuadruple presición
   \end{minted}
  \item Sin embargo, en muchos casos es útil predefinir la clase kind al cambiar de tipo de real variando el valor de np, como por ejemplo 
  \vspace{0.2cm}
  \item []
   \begin{minted}[linenos,autogobble]{fortran}
    INTEGER, PARAMETER :: np=16         !np = 4,8 o 16
    REAL (kind=np) :: X = 2.e-10_np     !e =  e, d o q
   \end{minted}
 \end{itemize}
\end{frame}


\begin{frame}[fragile]{Complejos}
\onslide<0->\textbf{Tipos de complejos}
 \begin{itemize}[<+(1)->]
  \item La representación de números complejos se declara con COMPLEX, e igualmente que los reales, puede presentar una presición simple, doble o cuádruple.
  \vspace{0.2cm}
  \item []
   \begin{minted}[linenos,autogobble]{fortran}
    INTEGER :: re = 25
    REAL(kind= 4) :: im04 = 3.141592
    REAL(kind= 8) :: im08 = 3.141592
    REAL(kind=10) :: im10 = 3.141592
    REAL(kind=16) :: im16 = 3.141592
    COMPLEX(kind=16) :: z16 = (25, 3.141592)
   \end{minted}
  \rightline {\textit{Véase complex.f95}}
  \item La notación científica y las declaraciones empleando kind siguen las mismas reglas.
 \end{itemize}
\end{frame}