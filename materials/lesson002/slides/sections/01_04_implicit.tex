%-----------------------------------------------------------------------------80
% CONTENT
%-----------------------------------------------------------------------------80
%Definición-------------------------------------------------------------------80

\subsection{Cláusula IMPLICIT}

\begin{frame}[fragile]{La cláusula Implicit}
 \begin{itemize}[<+(0)->]
  \item Si existen variables que no han sido definidas, el tipo de variable depende de la letra inicial. Por lo cual
  \item[-] i, j, k, l, m, n representan variables enteras.
  \item[-] Las demás letras representan variables reales de presición simple.
  \item El carácter implicito puede ser modificado empleando la instrucción IMPLICIT bajo la siguiente sintaxis:\\
  \item Debido a que el compilador puede reconocer variables por defecto, se recomienda emplear la instrucción IMPLICIT NONE, especificando todas las variables y evitando errores con el código fuente.
 \end{itemize}
\end{frame}
