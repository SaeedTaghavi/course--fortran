%-----------------------------------------------------------------------------80
% SECTION TITLE|
%-----------------------------------------------------------------------------80

\section{Variables y tipos de datos}  

%-----------------------------------------------------------------------------80
% CONTENT
%-----------------------------------------------------------------------------80
%Definición-------------------------------------------------------------------80

\subsection{Variables y tipos de datos}

\begin{frame}[fragile]{Variables y tipos de datos}
 \begin{itemize}[<+(0)->]
  \item Un lenguaje de programación permite identificar los datos que se manipulan y almacenan en grandes cantidades en un ordenador.
 \end{itemize}

\onslide<2->\textbf{Variables}
 \begin{itemize}[<+(1)->]
  \item Una variable es un objeto que representa un tipo de dato, suceptible de modificarse, nombrado por cadenas de caracteres.
 \end{itemize}
\end{frame}

\begin{frame}[fragile]{Variables y tipos de datos}
\onslide<0->\textbf{Tipos de datos}
 \begin{enumerate}[<+(1)->]
  \item \textbf{character:} cadena de uno o varios caracteres.
  \item \textbf{integer:} números enteros, positivos o negativos.
  \item \textbf{logical:} valores lógicos o booleanos, es decir, toman uno de los dos valores, .true. (verdadero) o .false. (falso).
  \item \textbf{real:} números reales, positivos o negativos.
  \item \textbf{complex:} números complejos, compuestos de una parte real y otra imaginaria, ambas de tipo real.
 \end{enumerate}
 \end{frame}

