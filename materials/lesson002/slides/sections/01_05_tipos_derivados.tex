%-----------------------------------------------------------------------------80
% CONTENT
%-----------------------------------------------------------------------------80
%Definición-------------------------------------------------------------------80

\subsection{Tipos derivados}

\begin{frame}[fragile]{La cláusula Implicit}
 \begin{itemize}[<+(0)->]
  \item Disponibles a partir de Fortran90 (solo se podía trabajar con INTEGER, REAL, LOGICAL Y CHARACTER).
  \item Son expresiones derivadas de datos, es decir, creadas a parir de los tipos intrínsecos.
  \item La sintaxis para declarar un nuevo tipo es de la forma
  \item [] 
   \begin{minted}[linenos,autogobble]{fortran}
    type <Nombre del nuevo tipo>
         <Instruccion declaracion variable(s)>
         <Instruccion declaracion variable(s)>
         :
         <Instruccion declaracion variable(s)>
    end type
   \end{minted}
  \item La sintaxis para declarar una o más variables de un tipo derivado se expresa de la forma\\
      \centering type (<nombre tipo>),[<atributos>]::<var\_1,var\_2...>=[<valor>]
 \end{itemize}
\end{frame}