%-----------------------------------------------------------------------------80
% CONTENT
%-----------------------------------------------------------------------------80
%Definición-------------------------------------------------------------------80

\subsection{Operaciones elementales}


\begin{frame}[fragile]{Funciones intrínsecas \textit{built in functions}}  
 \begin{itemize}[<+(0)->]
  \item Funciones predefinidas independientes del compilador. 
    \item Una función intrínseca monoargumental se expresa de la forma \\
      \centering <función> (<argumento>)\\ 
  \item Las funciones intrínsecas pueden no tener argumento o tener varios.
  \item Las funciones biargumentales más destacadas son la función CMPLX y la función MOD.
  \item [-] CMPLX asigna un valor complejo a partir de valores reales. 
   \begin{minted}[linenos,autogobble]{fortran}
    !Sea z = (x,y) una variable compleja
    CMPLX (<expres_1>, <expres_2>) !expres son de tipo REAL
   \end{minted}
  \vspace{0.2cm}
  \item [-] MOD es el valor que corresponde al resto de una división, es decir\\
      \centering $n = n/m + MOD(n, m) $
   \begin{minted}[linenos,autogobble]{fortran}
    !Sea n de tipo INTEGER o REAL y m del mismo tipo y distinto de 0
    MOD (<expres_1>, <expres_2>) !expres son de tipo INTEGER o REAL
   \end{minted}  
 \end{itemize}
\end{frame}

\begin{frame}[fragile]{Funciones intrínsecas \textit{built in functions}}  
    \begin{table}[]
    \centering
    \label{Tabla_funcionesintr}
    \resizebox{10.5cm}{!} {
    \begin{tabular}{|c|c|c|c|}
    \hline
    Función         & Argumento                     & Resultado         & Descripción                                                       \\ \hline
    abs             & real, complex, integer        & real, integer     & |x|, |z|, |n|                                                     \\ \hline
    sqrt            & real, complex                 & real, complex     & ($\sqrt{x}$, x $\geq$ 0), ($\sqrt{x}$, z $\in \mathbb{C}$)        \\ \hline
    int             & real                          & integer           & Parte entera de un real x                                         \\ \hline
    fraccion        & real                          & real              & Parte fraccional de un real x                                     \\ \hline
    real            & complex                       & real              & $\mathbb{R}$e(z), z $\in \mathbb{C}$                              \\ \hline
    aimag           & complex                       & real              & $\mathbb{I}$m(z), z $\in \mathbb{C}$                              \\ \hline              
    conjg           & complex                       & complex           & $\bar{z}$, z $\in \mathbb{C}$                                     \\ \hline 
    cos             & real complex                  & real complex      & $(\cos x, x \in \mathbb{R})$, $(\cos z, z \in \mathbb{C})$        \\ \hline 
    sin             & real complex                  & real complex      & $(\sin x, x \in \mathbb{R})$, $(\cos z, z \in \mathbb{C})$        \\ \hline 
    tan             & real complex                  & real complex      & $(\tan x, x \in \mathbb{R})$, $(\tan z, z \in \mathbb{C})$        \\ \hline 
    acos            & real complex                  & real complex      & $(\arccos x, x \in \mathbb{R})$, $(\arccos z, z \in \mathbb{C})$  \\ \hline 
    asin            & real complex                  & real complex      & $(\arcsin x, x \in \mathbb{R})$, $(\arcsin z, z \in \mathbb{C})$  \\ \hline 
    atan            & real complex                  & real complex      & $(\arctan x, x \in \mathbb{R})$, $(\arctan z, z \in \mathbb{C})$     \\ \hline 
    exp             & real complex                  & real complex      & $(\exp x, x \in \mathbb{R})$, $(\exp z, z \in \mathbb{C})$        \\ \hline 
    log             & real complex                  & real complex      & $(\log x, x > 0)$, $(\log z, z \in \mathbb{C}, z \neq 0)$          \\ \hline 
    log10           & real                          & real              & $(\log_{10} x, x > 0$                                             \\ \hline 
    \end{tabular}}
    \caption*{Funciones intrínsecas más relavantes}
    \end{table}
\end{frame}


\begin{frame}[fragile]{Operaciones con caracteres}
     \begin{itemize}[<+(0)->] 
    \item La asignación del tipo CHARACTER es de la forma \\
      \centering <variable> = <expresión>\\ 
    \item [] donde <variable> y <expresión> son de tipo CHARACTER y pueden tener longitud ($len = n$) o ($len = m$) respectivamente; $n, m \in \mathbb{Z}^{+}$ 
    \vspace{0.2cm}
    \item [-] Si $n \leq m$, se asigna a la <variable> los n primeros caracteres de la <expresión> de izquierda a derecha, eliminando la diferencia $m-n$.
    \item [-] Si $n > m$, se asigna a la <variable> de izquierda a derecha la cadena de caracteres de <expresión>, completando los últimos $n-m$ caracteres de la derecha con espacios. 
     \end{itemize}

\end{frame}

\begin{frame}[fragile]{Operaciones con caracteres}
\textbf{Operadores binarios}
 \begin{itemize}[<+(1)->]  
    \item Concatenación: dado por el operador //.
    \item Comparación: dado por los operadores == y /=.
 \end{itemize}

\onslide<3->\textbf{Partes de cadenas de caracteres}
 \begin{itemize}[<+(2)->]  
    \item Si <expresión> representa una cadena de caracteres $c_{1}...c_{k}...c_{l}...c_{n}$ de $n$ caracteres, con $1 \leq k \leq l \leq n$, podemos obtener lo siguiente:
     \begin{minted}[linenos,autogobble]{fortran}
     <expresión> (k:l) !cadena "c_k...c_l"
     <expresión> (:l) !cadena "c_1...c_k...c_l"
     <expresión> (k:) !cadena "c_k...c_l...c_n"
     \end{minted}  
 \end{itemize}
\end{frame}

\begin{frame}[fragile]{Operaciones con caracteres}
\onslide<0->\textbf{Operaciones sobre cadenas de caracteres}
 \begin{itemize}[<+(2)->]
  \item La sintaxis para instrucciones de tipo CHARACTER es \\
    \centering <instrucción> (<expresión>)
  \item []
  \begin{table}[]
    \centering
    \label{Tabla_funcionesintr}
    \resizebox{10.5cm}{!} {
    \begin{tabular}{|c|c|p{6cm}|}
    \hline
    Instrucción  & Resultado  & Descripción                                                                                                    \\ \hline
    len          & integer    & Define la longitud una cadena de caracteres                                                                    \\ \hline
    trim         & character  & Suprime los epacios del final de una cadena                                                                    \\ \hline
    adjustl      & character  & Si hay espacios al inicio de una cadena, los suprime desplazando el resto de la cadena a la izquierda.        \\ \hline
    \end{tabular}}
    \caption*{Instrucciones de cadenas de caracteres}
    \end{table}
 \end{itemize}
\end{frame}


\begin{frame}[fragile]{Operaciones con caracteres}
 \onslide<0->\textbf{La instrucción INDEX}
  \begin{itemize}[<+(1)->]
   \item La instrucción INDEX proporciona como resultado un valor tipo INTEGER a partir de una cadena de tipo CHARACTER.
   \item La sintaxis es de la forma \\
    \centering INDEX (<expresión\_1>, <expresión\_2>).

   \item Si <expresión\_2> es parte de <expresión\_1> entonces indica su primera posición; sino el valor es 0.
   \item []
    \begin{minted}[linenos,autogobble]{fortran}
     INDEX ('Fortran','tra') ! INDEX dará 4   
    \end{minted} 
  \end{itemize}
\end{frame}


