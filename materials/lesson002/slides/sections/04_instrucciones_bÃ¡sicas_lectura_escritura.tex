%-----------------------------------------------------------------------------80
% SECTION TITLE|
%-----------------------------------------------------------------------------80

\section{Instrucciones básicas de lectura y escritura de datos}  

%-----------------------------------------------------------------------------80
% CONTENT
%-----------------------------------------------------------------------------80
%Definición-------------------------------------------------------------------80

\subsection{Escritura sobre pantalla y entrada de datos}

\begin{frame}[fragile]{Lectura y escritura de datos}
\onslide<0->\textbf{Escritura sobre la pantalla}
 \begin{itemize}[<+(1)->]
  \item Se utilizan dos instrucciones equivalentes: PRINT y WRITE.
  \item La sintaxis es de la forma 
  \item []
   \begin{minted}[linenos,autogobble]{fortran}
    print*, <expresion_1>, ..., <expresion_n>
    write(*,*) <expresion_1>, ..., <expresion_n>
    write(6,*) <expresion_1>, ..., <expresion_n>
   \end{minted}
  \item El caracter * permite que la computadora predetermine un formato a los datos.
  \item Cada expresión debe ser de tipo INTEGER, REAL, COMPLEX, CHARACTER o LOGICAL.
 \end{itemize}
\end{frame}

\begin{frame}[fragile]{Lectura y escritura de datos}
 \onslide<0->\textbf{Lectura de datos utlizando el teclado? (entrada de dato por el usuario) }
 \begin{itemize}[<+(1)->]
  \item Se utiliza la intrucción READ
  \item La sintaxis es de la forma
  \vspace{0.2cm}
  \item []
   \begin{minted}[linenos,autogobble]{fortran}
    READ*, <variable_1>, ..., <variable_n>
    READ(*,*) <variable_1>, ..., <variable_n>
    READ(5,*) <variable_1>, ..., <variable_n>
   \end{minted}
  \item Cada variable debe ser de tipo INTEGER, REAL, COMPLEX, CHARACTER o LOGICAL. 
  \item En caso de las cadenas de caracteres es necesario delimitarlas con comillas.
 \end{itemize}
\end{frame}

\begin{frame}[fragile]{Lectura y escritura de datos}
 \onslide<0-> \textbf{Lectura de datos utlizando el teclado? (entrada de dato por el usuario) 2003}
 \begin{itemize}[<+(1)->]
  \item 
 \end{itemize}   
\end{frame}
