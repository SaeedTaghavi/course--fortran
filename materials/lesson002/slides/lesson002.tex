%%----------------------------------------------------------------------------80
%% Preamble
%%----------------------------------------------------------------------------80
\documentclass[10pt,aspectratio=96]{beamer}
\usepackage[some]{background}
\usepackage[T1]{fontenc}
\usepackage{amsmath}
\usepackage{appendixnumberbeamer}
\usepackage{array}
\usepackage{booktabs}
\usepackage{colortbl}
\usepackage{fontawesome}
\usepackage{geometry}
\usepackage{graphicx}
\usepackage{hologo}
\usepackage{lipsum}
\usepackage{multirow}
\usepackage{pdflscape}
\usepackage{pgfcalendar}
\usepackage{pgfplots}
\usepackage{polyglossia}
\usepackage{tabularx}
\usepackage{xcolor-material}
\usepackage{xcolor}
\usepackage{xspace}

% \usepackage{float}
\usepackage{caption}
\usepackage[outputdir=build]{minted}
\usepackage{hyperref}


%%----------------------------------------------------------------------------80
%% Settings
%%----------------------------------------------------------------------------80
% Poligrossia pacakge settings -----------------------------------------------80
\setmainlanguage{spanish}


% Hyperref package settings --------------------------------------------------80
\hypersetup{
  pdftitle={Programaci\'{o}n en FORTRAN - Lecci\'{o}n 1},
  pdfauthor={Mart\'{i}n Josemar\'{i}a Vuelta Rojas},
  pdfpagelayout=OneColumn,
  pdfnewwindow=true,
  pdfdisplaydoctitle=true,
  pdfstartview=XYZ,
  plainpages=false,
  unicode=true,
  bookmarksnumbered=true,
  bookmarksopen=true,
  bookmarksopenlevel=3,
  breaklinks=true,
  colorlinks=true,
  pdfborder={0 0 0}
}


% Caption package settings ---------------------------------------------------80
\captionsetup[figure]{labelfont=bf,justification=centering}


% PGFPlots package settings --------------------------------------------------80
\pgfplotsset{compat=1.14}
\usepgfplotslibrary{dateplot}
\usepgfplotslibrary{groupplots}


% Beamer package settings ----------------------------------------------------80
\usetheme{metropolis}


% Minted package settings ----------------------------------------------------80
\usemintedstyle{manni}
\setminted{
  fontsize=\scriptsize,
  baselinestretch=1.15
}

%%----------------------------------------------------------------------------80
%% Customizations
%%----------------------------------------------------------------------------80

%----------FONDO DE CÓDIGO MINTED----------------%
\newminted[mintedbash]{bash}{linenos=true, autogobble=true, texcl=true, bgcolor=grey-300, fontsize=auto }

%---------------------------------------------%


%%----------------------------------------------------------------------------80
%% Custom commands definitions
%%----------------------------------------------------------------------------80

\makeatletter
\definecolor{green-600}{HTML}{43A047}
\definecolor{grey-300}{HTML}{E0E0E0}

%----------FONDO DE CÓDIGO MINTED----------------%
\renewenvironment{minted@colorbg}[1]
 {\def\minted@bgcol{#1}%
  \noindent
  \begin{lrbox}{\minted@bgbox}
  \begin{minipage}{\linewidth-2\fboxsep}}
 {\end{minipage}%
  \end{lrbox}%
  \setlength{\topsep}{\bigskipamount}% set the vertical space
  \trivlist\item\relax % ensure going to a new line
  \colorbox{\minted@bgcol}{\usebox{\minted@bgbox}}%
  \endtrivlist % close the trivlist
 }
 %-----------------------------------------------%

 \makeatother


%%----------------------------------------------------------------------------80
%% Document
%%----------------------------------------------------------------------------80
% Global document settings ---------------------------------------------------80
% Document body --------------------------------------------------------------80
\begin{document}
  % -*- BEGIN: Title Frame
  \title{Programación en FORTRAN}
  \subtitle{
    Nivel Básico - Sesión 2
  }
  \date{\today}
  \author{Martin Josemaría Vuelta Rojas}
  \institute{SoftButterfly}
  \maketitle
  % -*- END: Title Frame

  % -*- BEGIN: TOC
  \begin{frame}{Contenido}
    \setbeamertemplate{section in toc}[sections numbered]
    \tableofcontents[hideallsubsections]
  \end{frame}
  % -*- END: TOC

  %-----------------------------------------------------------------------------80
% SECTION TITLE|
%-----------------------------------------------------------------------------80

\section{Variables y tipos de datos}  

%-----------------------------------------------------------------------------80
% CONTENT
%-----------------------------------------------------------------------------80
%Definición-------------------------------------------------------------------80

\subsection{Variables y tipos de datos}

\begin{frame}[fragile]{Variables y tipos de datos}
 \begin{itemize}[<+(0)->]
  \item Un lenguaje de programación permite identificar los datos que se manipulan y almacenan en grandes cantidades en un ordenador.
 \end{itemize}
 \vspace{0.2cm}
\onslide<2->\textbf{Variables}
 \begin{itemize}[<+(1)->]
  \item Una variable es un objeto que representa un tipo de dato, suceptible de modificarse, nombrado por cadenas de caracteres.
 \end{itemize}
\end{frame}

\begin{frame}[fragile]{Variables y tipos de datos}
\onslide<0->\textbf{Tipos de datos}
 \begin{enumerate}[<+(1)->]
  \item \textbf{character:} cadena de uno o varios caracteres.
  \item \textbf{integer:} números enteros, positivos o negativos.
  \item \textbf{logical:} valores lógicos o booleanos, es decir, toman uno de los dos valores, .true. (verdadero) o .false. (falso).
  \item \textbf{real:} números reales, positivos o negativos.
  \item \textbf{complex:} números complejos, compuestos de una parte real y otra imaginaria, ambas de tipo real.
 \end{enumerate}
 \end{frame}


  %-----------------------------------------------------------------------------80
% CONTENT
%-----------------------------------------------------------------------------80
%Definición-------------------------------------------------------------------80

\subsection{Variables y constantes}

\begin{frame}[fragile]{Variables y constantes}
\onslide<0->\textbf{Declaración de variables}
 \begin{itemize}[<+(1)->]
  \item La declaración de una o más variables del mismo tipo está dada por la sintaxis\\ 
   \centering <tipo> ,[<atributo(s)>] [::] <variable(s)> [=<valor>]
  \item Algunos atributos son:\\ 
  parameter, save, pointer, target, allocatable, dimension, public, private, external, intrinsic, optional.
  \vspace{6pt}
  \item []
   \begin{minted}[linenos,autogobble]{fortran}
    CHARACTER(len= 4), PARAMETER :: prompt = ">>> "
    CHARACTER(len= *), PARAMETER :: message = "Ingresa tu primer nombre [máx 20 car]:"
   \end{minted}
  \item[] \rightline {\textit{Véase strings.f95}}
 \end{itemize}
\end{frame}

\begin{frame}[fragile]{Variables y constantes}
\onslide<0->\textbf{Declaración de constantes}
 \begin{itemize}[<+(1)->]
  \item Si se requiere que una variable que tome un valor definido no suceptible de cambio, se utiliza el atributo parameter. \\ 
   \begin{minted}[linenos,autogobble]{fortran}
    CHARACTER, PARAMETER :: NewLine = CHAR(10)
   \end{minted}
   \item[] \rightline {\textit{Véase strings.f95}}
   \vspace{-3pt}
   \item Las variables pueden ser definidas en función de constantes mediante el atributo parameter.
 \end{itemize}
\end{frame}

  %-----------------------------------------------------------------------------80
% CONTENT
%-----------------------------------------------------------------------------80
%Definición-------------------------------------------------------------------80

\subsection{Cadenas de caracteres y valores lógicos}

\begin{frame}[fragile]{Cadenas de caracteres y valores lógicos}
\onslide<0->\textbf{Declaración de cadenas de caracteres}
 \begin{itemize}[<+(1)->]
  \item La declaración de una variable de tipo character está dada por la sintaxis\\ 
   \centering CHARACTER[(len=<long>)],[<atributos>][::]<variables>[=<valor>]
  \vspace{0.2cm}
  \item []
   \begin{minted}[linenos,autogobble]{fortran}
    CHARACTER(kind=ascii, len=26) :: Alphabet
    CHARACTER(kind= ucs4, len=30) :: HelloWorld
   \end{minted}
 \rightline {\textit{Véase kind\_character.f95}}
 \end{itemize}

\onslide<4->\textbf{Declaración de valores lógicos}
   \begin{itemize}[<+(2)->]
  \item La declaración de una variable lógica está dada por\\ 
   \centering LOGICAL <variable(s)>
 \end{itemize}
\end{frame}


  %-----------------------------------------------------------------------------80
% CONTENT
%-----------------------------------------------------------------------------80
%Definición-------------------------------------------------------------------80

\subsection{Enteros, reales y complejos}

\begin{frame}[fragile]{Enteros}
\onslide<0->\textbf{Tipos de enteros}
 \begin{itemize}[<+(1)->]
  \item La representación de valores enteros se declara con INTEGER.
  \item Los valores pueden ser guardados usualmente con presición simple, doble o cuádruple.  
  \vspace{0.2cm}
  \item []
   \begin{minted}[linenos,autogobble]{fortran} 
    INTEGER, PARAMETER :: K02 = SELECTED_INT_KIND(2)
    INTEGER, PARAMETER :: K04 = SELECTED_INT_KIND(4)
    INTEGER, PARAMETER :: K08 = SELECTED_INT_KIND(8)
    INTEGER, PARAMETER :: K16 = SELECTED_INT_KIND(16)

    INTEGER(kind=K02) :: I02
    INTEGER(kind=K04) :: I04
    INTEGER(kind=K08) :: I08
    INTEGER(kind=K16) :: I16
   \end{minted}
  \rightline {\textit{Véase kind\_integers.f95}}
 \end{itemize}
\end{frame}

\begin{frame}[fragile]{Reales}
\onslide<0->\textbf{Tipos de reales}
 \begin{itemize}[<+(1)->]
  \item La representación de número reales se declara con REAL y puede ser de presición estándar o simple presición (sp) y de presición superior, doble (dp) o cuádruple (qp) presición en adelante.
  \item La sintaxis para el tipo real es\\ 
   \centering REAL (kind=<np>)
  \vspace{0.2cm}
  \item []
   \begin{minted}[linenos,autogobble]{fortran}
    REAL(kind=p04) :: X04
    REAL(kind=p08) :: X08
    REAL(kind=p16) :: X16
    REAL(kind=p32) :: X32
   \end{minted}
 \rightline {\textit{Véase kind\_real.f95}}
 \end{itemize}
\end{frame}


\begin{frame}[fragile]{Reales}
 \begin{itemize}[<+(0)->]
  \item La notación científica para los reales viene dada por los identificadores "e" (sp), "d" (dp) y "q" (qp).  
  \vspace{0.2cm}
  \item []
   \begin{minted}[linenos,autogobble]{fortran}
    REAL(kind=4)  :: x = 2.e0       !simple presición
    REAL(kind=8)  :: y = 4.d-6      !doble presición
    REAL(kind=16) :: z = -8.q-1000  !cuadruple presición
   \end{minted}
  \item Sin embargo, en muchos casos es útil predefinir la clase kind al cambiar de tipo de real variando el valor de np, como por ejemplo 
  \vspace{0.2cm}
  \item []
   \begin{minted}[linenos,autogobble]{fortran}
    INTEGER, PARAMETER :: np=16         !np = 4,8 o 16
    REAL (kind=np) :: X = 2.e-10_np     !e =  e, d o q
   \end{minted}
 \end{itemize}
\end{frame}


\begin{frame}[fragile]{Complejos}
\onslide<0->\textbf{Tipos de complejos}
 \begin{itemize}[<+(1)->]
  \item La representación de números complejos se declara con COMPLEX, e igualmente que los reales, puede presentar una presición simple, doble o cuádruple.
  \vspace{0.2cm}
  \item []
   \begin{minted}[linenos,autogobble]{fortran}
    INTEGER :: re = 25
    REAL(kind= 4) :: im04 = 3.141592
    REAL(kind= 8) :: im08 = 3.141592
    REAL(kind=10) :: im10 = 3.141592
    REAL(kind=16) :: im16 = 3.141592
    COMPLEX(kind=16) :: z16 = (25, 3.141592)
   \end{minted}
  \rightline {\textit{Véase complex.f95}}
  \item La notación científica y las declaraciones empleando kind siguen las mismas reglas.
 \end{itemize}
\end{frame}
  \input{sections/02_operaciones_elementales}
  \input{sections/03_estructuras_control}
  \input{sections/04_instrucciones_básicas_lectura_escritura}
  \input{sections/05_proyecto_euler}
\end{document}