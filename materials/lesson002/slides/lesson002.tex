%%----------------------------------------------------------------------------80
%% Preamble
%%----------------------------------------------------------------------------80
\documentclass[10pt,aspectratio=96]{beamer}
\usepackage[some]{background}
\usepackage[T1]{fontenc}
\usepackage{amsmath}
\usepackage{appendixnumberbeamer}
\usepackage{array}
\usepackage{booktabs}
\usepackage{colortbl}
\usepackage{fontawesome}
\usepackage{geometry}
\usepackage{graphicx}
\usepackage{hologo}
\usepackage{lipsum}
\usepackage{multirow}
\usepackage{pdflscape}
\usepackage{pgfcalendar}
\usepackage{pgfplots}
\usepackage{polyglossia}
\usepackage{tabularx}
\usepackage{xcolor-material}
\usepackage{xcolor}
\usepackage{xspace}

% \usepackage{float}
\usepackage{caption}
\usepackage[outputdir=build]{minted}
\usepackage{hyperref}


%%----------------------------------------------------------------------------80
%% Settings
%%----------------------------------------------------------------------------80
% Poligrossia pacakge settings -----------------------------------------------80
\setmainlanguage{spanish}


% Hyperref package settings --------------------------------------------------80
\hypersetup{
  pdftitle={Programaci\'{o}n en FORTRAN - Lecci\'{o}n 1},
  pdfauthor={Mart\'{i}n Josemar\'{i}a Vuelta Rojas},
  pdfpagelayout=OneColumn,
  pdfnewwindow=true,
  pdfdisplaydoctitle=true,
  pdfstartview=XYZ,
  plainpages=false,
  unicode=true,
  bookmarksnumbered=true,
  bookmarksopen=true,
  bookmarksopenlevel=3,
  breaklinks=true,
  colorlinks=true,
  pdfborder={0 0 0}
}


% Caption package settings ---------------------------------------------------80
\captionsetup[figure]{labelfont=bf,justification=centering}


% PGFPlots package settings --------------------------------------------------80
\pgfplotsset{compat=1.14}
\usepgfplotslibrary{dateplot}
\usepgfplotslibrary{groupplots}


% Beamer package settings ----------------------------------------------------80
\usetheme{metropolis}


% Minted package settings ----------------------------------------------------80
\usemintedstyle{manni}
\setminted{
  fontsize=\scriptsize,
  baselinestretch=1.15
}

%%----------------------------------------------------------------------------80
%% Customizations
%%----------------------------------------------------------------------------80

%----------FONDO DE CÓDIGO MINTED----------------%
\newminted[mintedbash]{bash}{linenos=true, autogobble=true, texcl=true, bgcolor=grey-300, fontsize=auto }

%---------------------------------------------%


%%----------------------------------------------------------------------------80
%% Custom commands definitions
%%----------------------------------------------------------------------------80

\makeatletter
\definecolor{green-600}{HTML}{43A047}
\definecolor{grey-300}{HTML}{E0E0E0}

%----------FONDO DE CÓDIGO MINTED----------------%
\renewenvironment{minted@colorbg}[1]
 {\def\minted@bgcol{#1}%
  \noindent
  \begin{lrbox}{\minted@bgbox}
  \begin{minipage}{\linewidth-2\fboxsep}}
 {\end{minipage}%
  \end{lrbox}%
  \setlength{\topsep}{\bigskipamount}% set the vertical space
  \trivlist\item\relax % ensure going to a new line
  \colorbox{\minted@bgcol}{\usebox{\minted@bgbox}}%
  \endtrivlist % close the trivlist
 }
 %-----------------------------------------------%

 \makeatother


%%----------------------------------------------------------------------------80
%% Document
%%----------------------------------------------------------------------------80
% Global document settings ---------------------------------------------------80
% Document body --------------------------------------------------------------80
\begin{document}
  % -*- BEGIN: Title Frame
  \title{Programación en FORTRAN}
  \subtitle{
    Nivel Básico - Sesión 2
  }
  \date{\today}
  \author{Martin Josemaría Vuelta Rojas}
  \institute{SoftButterfly}
  \maketitle
  % -*- END: Title Frame

  % -*- BEGIN: TOC
  \begin{frame}{Contenido}
    \setbeamertemplate{section in toc}[sections numbered]
    \tableofcontents[hideallsubsections]
  \end{frame}
  % -*- END: TOC

  %-----------------------------------------------------------------------------80
% SECTION TITLE|
%-----------------------------------------------------------------------------80

\section{Variables y tipos de datos}  

%-----------------------------------------------------------------------------80
% CONTENT
%-----------------------------------------------------------------------------80
%Definición-------------------------------------------------------------------80

\subsection{Variables y tipos de datos}

\begin{frame}[fragile]{Variables y tipos de datos}
 \begin{itemize}[<+(0)->]
  \item Un lenguaje de programación permite identificar los datos que se manipulan y almacenan en grandes cantidades en un ordenador.
 \end{itemize}
 \vspace{0.2cm}
\onslide<2->\textbf{Variables}
 \begin{itemize}[<+(1)->]
  \item Una variable es un objeto que representa un tipo de dato, suceptible de modificarse, nombrado por cadenas de caracteres.
 \end{itemize}
\end{frame}

\begin{frame}[fragile]{Variables y tipos de datos}
\onslide<0->\textbf{Tipos de datos}
 \begin{enumerate}[<+(1)->]
  \item \textbf{character:} cadena de uno o varios caracteres.
  \item \textbf{integer:} números enteros, positivos o negativos.
  \item \textbf{logical:} valores lógicos o booleanos, es decir, toman uno de los dos valores, .true. (verdadero) o .false. (falso).
  \item \textbf{real:} números reales, positivos o negativos.
  \item \textbf{complex:} números complejos, compuestos de una parte real y otra imaginaria, ambas de tipo real.
 \end{enumerate}
 \end{frame}


  %-----------------------------------------------------------------------------80
% CONTENT
%-----------------------------------------------------------------------------80
%Definición-------------------------------------------------------------------80

\subsection{Variables y constantes}

\begin{frame}[fragile]{Variables y constantes}
\onslide<0->\textbf{Declaración de variables}
 \begin{itemize}[<+(1)->]
  \item La declaración de una o más variables del mismo tipo está dada por la sintaxis\\ 
   \centering <tipo> ,[<atributo(s)>] [::] <variable(s)> [=<valor>]
  \item Algunos atributos son:\\ 
  parameter, save, pointer, target, allocatable, dimension, public, private, external, intrinsic, optional.
  \vspace{6pt}
  \item []
   \begin{minted}[linenos,autogobble]{fortran}
    CHARACTER(len= 4), PARAMETER :: prompt = ">>> "
    CHARACTER(len= *), PARAMETER :: message = "Ingresa tu primer nombre [máx 20 car]:"
   \end{minted}
  \item[] \rightline {\textit{Véase strings.f95}}
 \end{itemize}
\end{frame}

\begin{frame}[fragile]{Variables y constantes}
\onslide<0->\textbf{Declaración de constantes}
 \begin{itemize}[<+(1)->]
  \item Si se requiere que una variable que tome un valor definido no suceptible de cambio, se utiliza el atributo parameter. \\ 
   \begin{minted}[linenos,autogobble]{fortran}
    CHARACTER, PARAMETER :: NewLine = CHAR(10)
   \end{minted}
   \item[] \rightline {\textit{Véase strings.f95}}
   \vspace{-3pt}
   \item Las variables pueden ser definidas en función de constantes mediante el atributo parameter.
 \end{itemize}
\end{frame}

  %-----------------------------------------------------------------------------80
% CONTENT
%-----------------------------------------------------------------------------80
%Definición-------------------------------------------------------------------80

\subsection{Cadenas de caracteres y valores lógicos}

\begin{frame}[fragile]{Cadenas de caracteres y valores lógicos}
\onslide<0->\textbf{Declaración de cadenas de caracteres}
 \begin{itemize}[<+(1)->]
  \item La declaración de una variable de tipo character está dada por la sintaxis\\ 
   \centering CHARACTER[(len=<long>)],[<atributos>][::]<variables>[=<valor>]
  \vspace{0.2cm}
  \item []
   \begin{minted}[linenos,autogobble]{fortran}
    CHARACTER(kind=ascii, len=26) :: Alphabet
    CHARACTER(kind= ucs4, len=30) :: HelloWorld
   \end{minted}
 \rightline {\textit{Véase kind\_character.f95}}
 \end{itemize}

\onslide<4->\textbf{Declaración de valores lógicos}
   \begin{itemize}[<+(2)->]
  \item La declaración de una variable lógica está dada por\\ 
   \centering LOGICAL <variable(s)>
 \end{itemize}
\end{frame}


  %-----------------------------------------------------------------------------80
% CONTENT
%-----------------------------------------------------------------------------80
%Definición-------------------------------------------------------------------80

\subsection{Enteros, reales y complejos}

\begin{frame}[fragile]{Enteros}
\onslide<0->\textbf{Tipos de enteros}
 \begin{itemize}[<+(1)->]
  \item La representación de valores enteros se declara con INTEGER.
  \item Los valores pueden ser guardados usualmente con presición simple, doble o cuádruple.  
  \vspace{0.2cm}
  \item []
   \begin{minted}[linenos,autogobble]{fortran} 
    INTEGER, PARAMETER :: K02 = SELECTED_INT_KIND(2)
    INTEGER, PARAMETER :: K04 = SELECTED_INT_KIND(4)
    INTEGER, PARAMETER :: K08 = SELECTED_INT_KIND(8)
    INTEGER, PARAMETER :: K16 = SELECTED_INT_KIND(16)

    INTEGER(kind=K02) :: I02
    INTEGER(kind=K04) :: I04
    INTEGER(kind=K08) :: I08
    INTEGER(kind=K16) :: I16
   \end{minted}
  \rightline {\textit{Véase kind\_integers.f95}}
 \end{itemize}
\end{frame}

\begin{frame}[fragile]{Reales}
\onslide<0->\textbf{Tipos de reales}
 \begin{itemize}[<+(1)->]
  \item La representación de número reales se declara con REAL y puede ser de presición estándar o simple presición (sp) y de presición superior, doble (dp) o cuádruple (qp) presición en adelante.
  \item La sintaxis para el tipo real es\\ 
   \centering REAL (kind=<np>)
  \vspace{0.2cm}
  \item []
   \begin{minted}[linenos,autogobble]{fortran}
    REAL(kind=p04) :: X04
    REAL(kind=p08) :: X08
    REAL(kind=p16) :: X16
    REAL(kind=p32) :: X32
   \end{minted}
 \rightline {\textit{Véase kind\_real.f95}}
 \end{itemize}
\end{frame}


\begin{frame}[fragile]{Reales}
 \begin{itemize}[<+(0)->]
  \item La notación científica para los reales viene dada por los identificadores "e" (sp), "d" (dp) y "q" (qp).  
  \vspace{0.2cm}
  \item []
   \begin{minted}[linenos,autogobble]{fortran}
    REAL(kind=4)  :: x = 2.e0       !simple presición
    REAL(kind=8)  :: y = 4.d-6      !doble presición
    REAL(kind=16) :: z = -8.q-1000  !cuadruple presición
   \end{minted}
  \item Sin embargo, en muchos casos es útil predefinir la clase kind al cambiar de tipo de real variando el valor de np, como por ejemplo 
  \vspace{0.2cm}
  \item []
   \begin{minted}[linenos,autogobble]{fortran}
    INTEGER, PARAMETER :: np=16         !np = 4,8 o 16
    REAL (kind=np) :: X = 2.e-10_np     !e =  e, d o q
   \end{minted}
 \end{itemize}
\end{frame}


\begin{frame}[fragile]{Complejos}
\onslide<0->\textbf{Tipos de complejos}
 \begin{itemize}[<+(1)->]
  \item La representación de números complejos se declara con COMPLEX, e igualmente que los reales, puede presentar una presición simple, doble o cuádruple.
  \vspace{0.2cm}
  \item []
   \begin{minted}[linenos,autogobble]{fortran}
    INTEGER :: re = 25
    REAL(kind= 4) :: im04 = 3.141592
    REAL(kind= 8) :: im08 = 3.141592
    REAL(kind=10) :: im10 = 3.141592
    REAL(kind=16) :: im16 = 3.141592
    COMPLEX(kind=16) :: z16 = (25, 3.141592)
   \end{minted}
  \rightline {\textit{Véase complex.f95}}
  \item La notación científica y las declaraciones empleando kind siguen las mismas reglas.
 \end{itemize}
\end{frame}
  %-----------------------------------------------------------------------------80
% CONTENT
%-----------------------------------------------------------------------------80
%Definición-------------------------------------------------------------------80

\subsection{Cláusula IMPLICIT}

\begin{frame}[fragile]{La cláusula Implicit}
 \begin{itemize}[<+(0)->]
  \item Si existen variables que no han sido definidas, el tipo de variable depende de la letra inicial. Por lo cual...
  \item[-] i, j, k, l, m, n representan variables enteras.
  \item[-] las demás letras representan variables reales de presición simple.
  \item El carácter implicito puede ser modificado empleando la instrucción IMPLICIT bajo la siguiente sintaxis:\\
  \centering IMPLICIT <tipo> (<caracter(es)\_1>,...,<caracter(es)\_k>)
  \item Debido a que el compilador puede reconocer variables por defecto, se recomienda emplear la instrucción IMPLICIT NONE, especificando todas las variables y evitando errores con el código fuente.
 \end{itemize}
\end{frame}

  %-----------------------------------------------------------------------------80
% SECTION TITLE|
%-----------------------------------------------------------------------------80

\section{Operaciones elementales}  

%-----------------------------------------------------------------------------80
% CONTENT
%-----------------------------------------------------------------------------80
%Definición-------------------------------------------------------------------80

\subsection{Operaciones elementales}

\begin{frame}[fragile]{Operaciones elementales} 
\textbf{Reglas generales}
 \begin{itemize}[<+(1)->]
  \item Las operaciones siguen el orden tradicional de las operaciones matemáticas, es decir, primero los términos entre paréntesis, exponentes, multiplicación y adición.
  \item El uso del símbolo = en el lenguaje Fortran tiene el sentido de asignación mientras que en el uso matemático tiene sentido de igualdad. 
  \item La asignación de una variable tiene la sintaxis \\
      \centering <variable> = <expresión>\\
 \end{itemize}
\end{frame}

\begin{frame}[fragile]{Operaciones aritméticas}
 \begin{itemize}[<+(0)->]
  \item Estan permitidas las operaciones entre valores tipo INTEGER, REAL y COMPLEX.
  \item Las operaciones están dadas por adición (+), sustracción (-), multiplicación (*), división (/) y potenciación (**).
  \item Los operandos del mismo tipo y clase resultan en otro del mismo tipo y clase.
 \end{itemize}
\end{frame}

\begin{frame}[fragile]{Operaciones aritméticas}
\onslide<0->\textbf{Operaciones de tipo INTEGER}
 \begin{itemize}[<+(1)->]
  \item Las operaciones de tipo INTEGER manejan números enteros dentro de un rango en $\mathbb{Z}$.
  \item La división se obtiene con un resto; es decir\\
  \centering $\frac{x}{y} = z \Longleftrightarrow  |x| = |z| \cdot |y| + resto $
  \item La potenciación depende del tipo de variable del exponente. 
  $$
  x**n = \left\{ \begin{array}{lcc}
             \underbrace{x \ast x \ast \cdots \ast x}_\text{$n$ veces} &   si  & n > 0 \\
             \\ \frac{1}{x \ast \ast (-n)} &  si & n < 0 \\
             \\ 1 &  si  & x = 0 
             \end{array}
   \right.
  $$
 \end{itemize}
\end{frame}

\begin{frame}[fragile]{Operaciones aritméticas}
\onslide<0->\textbf{Conversión de tipos}
 \begin{itemize}[<+(1)->]
  \item Los enteros son convertidos en reales o complejos.
  \item Los reales son convertidos en complejos.
  \item Los reales o complejos son convertidos en la clase (kind) más alta.
  \item Al asignar valores (=), la parte derecha se evalúa en el tipo y clase correspondiente, luego es convertida al del tipo y clase de la variable al lado izquierdo.
  \vspace{0.2cm}
  \item[]
  \begin{minted}[linenos,autogobble]{fortran}
   INTEGER          :: n, m 
   REAL             :: a, b
   REAL(kind=8)     :: x, y
   COMPLEX          :: c
   COMPLEX(kind=8)  :: z
   :
   a = (x*(n**c))/z
   :
   \end{minted}
 \end{itemize}
\end{frame}


\begin{frame}[fragile]{Operaciones aritméticas}
\onslide<0->\textbf{Conversiones de tipo más significativas}
    \begin{table}[]
    \centering
    \label{Tabla_comparacion}
    \begin{tabular}{|l|l|l|l|}
    \hline
    Conversión      & Mecanismo de Conversión               \\ \hline
    x = n           & x = n                                 \\ \hline
    x = a           & x = a                                 \\ \hline
    n = x           & $  n = \left\{ \begin{array}{lcccc}
                 m &   si  & m \leq x \leq m+1   & y & x \geq 0 \\
             \\ -m &   si  & m \leq -x \leq m+1  & y & x < 0 
             \end{array}
   \right.$                                                 \\ \hline
    a = x           & a = round(x)                          \\ \hline
    a = c           & a = $\mathbb{Re}$(z)                   \\ \hline
    z = x           & z = (x, 0)                            \\ \hline              
    \end{tabular}
    
    \end{table}
\end{frame}
  %-----------------------------------------------------------------------------80
% CONTENT
%-----------------------------------------------------------------------------80
%Definición-------------------------------------------------------------------80

\subsection{Operaciones elementales}


\begin{frame}[fragile]{Operaciones elementales}
\onslide<0->\textbf{Operaciones de comparación}    
 \begin{itemize}[<+(0)->]
  \item La operación de comparación se expresa de la forma \\
      \centering <expresión\_1> <operador> <expresión\_2>
  \item Para variables de tipo COMPLEX solo son válidos los operadores == y /= .
  \item Los operadores de comparación son variables de tipo LOGICAL.
    \vspace{0.4cm}
  \item []
    \begin{table}[]
    \centering
    \label{Tabla_comparacion}
    \begin{tabular}{|l|l|l|l|}
    \hline
    Fortran 90  & Fortran 77    & Significado                  \\ \hline
    ==          & .eq.          & es igual a                   \\ \hline
    /=          & .ne.          & no es igual a                \\ \hline
    >           & .gt.          & es estrictamente mayor a     \\ \hline
    >=          & .ge.          & es mayor o igual a           \\ \hline
    <           & .lt.          & es estrictamente menor a     \\ \hline
    <=          & .le.          & es menor o igual a           \\ \hline              
    \end{tabular}
    \caption*{Operadores de comparación}
    \end{table}
  
 \end{itemize}
\end{frame}
  %-----------------------------------------------------------------------------80
% CONTENT
%-----------------------------------------------------------------------------80
%Definición-------------------------------------------------------------------80

\subsection{Operaciones elementales}


\begin{frame}[fragile]{Funciones intrínsecas \textit{built in functions}}  
 \begin{itemize}[<+(0)->]
  \item Funciones predefinidas independientes del compilador. 
    \item Una función intrínseca monoargumental se expresa de la forma \\
      \centering <función>(<argumento>)\\ 
  \item Las funciones intrínsecas pueden no tener argumento o tener varios.
  \item Las funciones biargumentales más destacadas son la función cmplx y la función mod.
  \item [-] cmplx asigna un valor complejo a partir de valores reales. 
   \begin{minted}[linenos,autogobble]{fortran}
    !Sea z = (x,y) una variable compleja
    cmplx (<expres_1>, <expres_2>) !expres son de tipo REAL
   \end{minted}
  \vspace{0.2cm}
  \item [-] mod es el valor que corresponde al resto de una división, es decir\\
      \centering $n = n/m + mod(n, m) $
   \begin{minted}[linenos,autogobble]{fortran}
    !Sea n de tipo INTEGER o REAL y m del mismo tipo y distinto de 0
    mod (<expres_1>, <expres_2>) !expres son de tipo INTEGER o REAL
   \end{minted}  
 \end{itemize}
\end{frame}

\begin{frame}[fragile]{Funciones intrínsecas \textit{built in functions}}  
    \begin{table}[]
    \centering
    \label{Tabla_funcionesintr}
    \resizebox{10.5cm}{!} {
    \begin{tabular}{|c|c|c|c|}
    \hline
    Función         & Argumento                     & Resultado         & Descripción                                                       \\ \hline
    abs             & real, complex, integer        & real, integer     & |x|, |z|, |n|                                                     \\ \hline
    sqrt            & real, complex                 & real, complex     & ($\sqrt{x}$, x $\geq$ 0), ($\sqrt{x}$, z $\in \mathbb{C}$)        \\ \hline
    int             & real                          & integer           & Parte entera de un real x                                         \\ \hline
    fraccion        & real                          & real              & Parte fraccional de un real x                                     \\ \hline
    real            & complex                       & real              & $\mathbb{R}$e(z), z $\in \mathbb{C}$                              \\ \hline
    aimag           & complex                       & real              & $\mathbb{I}$m(z), z $\in \mathbb{C}$                              \\ \hline              
    conjg           & complex                       & complex           & $\bar{z}$, z $\in \mathbb{C}$                                     \\ \hline 
    cos             & real complex                  & real complex      & $(\cos x, x \in \mathbb{R})$, $(\cos z, z \in \mathbb{C})$        \\ \hline 
    sin             & real complex                  & real complex      & $(\sin x, x \in \mathbb{R})$, $(\cos z, z \in \mathbb{C})$        \\ \hline 
    tan             & real complex                  & real complex      & $(\tan x, x \in \mathbb{R})$, $(\tan z, z \in \mathbb{C})$        \\ \hline 
    acos            & real complex                  & real complex      & $(\arccos x, x \in \mathbb{R})$, $(\arccos z, z \in \mathbb{C})$  \\ \hline 
    asin            & real complex                  & real complex      & $(\arcsin x, x \in \mathbb{R})$, $(\arcsin z, z \in \mathbb{C})$  \\ \hline 
    atan            & real complex                  & real complex      & $(\arctan x, x \in \mathbb{R})$, $(\arctan z, z \in \mathbb{C})$     \\ \hline 
    exp             & real complex                  & real complex      & $(\exp x, x \in \mathbb{R})$, $(\exp z, z \in \mathbb{C})$        \\ \hline 
    log             & real complex                  & real complex      & $(\log x, x > 0)$, $(\log z, z \in \mathbb{C}, z \neq 0)$          \\ \hline 
    log10           & real                          & real              & $(\log_{10} x, x > 0$                                             \\ \hline 
    \end{tabular}}
    \caption*{Funciones intrínsecas más relavantes}
    \end{table}
\end{frame}


\begin{frame}[fragile]{Operaciones con caracteres}
 \begin{itemize}[<+(0)->] 
    \item La asignación del tipo CHARACTER es de la forma \\
      \centering <variable> = <expresión>\\ 
    \item [] donde <variable> y <expresión> son de tipo CHARACTER y pueden tener longitud $len = n$ o $len = m$ respectivamente; $n, m \in \mathbb{Z}^{+}$ 
        \item [-] Si $n \leq m$, se asigna a la <variable> los n primeros caracteres de la <expresión> de izquierda a derecha, eliminando la diferencia $m-n$.
        \item [-] Si $n > m$, se asigna a la <variable> de izquierda a derecha la cadena de caracteres de <expresión>, completando los últimos $n-m$ caracteres de la derecha con espacios.
    \item asd
    \item asd
 \end{itemize}

\end{frame}
  \input{sections/03_estructuras_control}
  \input{sections/04_instrucciones_básicas_lectura_escritura}
  \input{sections/05_proyecto_euler}
\end{document}