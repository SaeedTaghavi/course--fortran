%%----------------------------------------------------------------------------80
%% Preamble
%%----------------------------------------------------------------------------80
\documentclass[12pt, twoside, a4paper, final]{article}
\usepackage[some]{background}
\usepackage[utf8]{inputenc}
\usepackage[spanish]{babel}
\usepackage[T1]{fontenc}
\usepackage{pgfcalendar}
\usepackage{pdflscape}
\usepackage{colortbl}
\usepackage{enumitem}
\usepackage{fancyhdr}
\usepackage{graphicx}
\usepackage{geometry}
\usepackage{hyperref}
\usepackage{multirow}
\usepackage{pgfplots}
\usepackage{tabularx}
\usepackage{titlesec}
\usepackage{amsmath}
\usepackage{caption}
\usepackage{charter}
\usepackage{hologo}
\usepackage{lipsum}
\usepackage{xcolor}
\usepackage{array}
\usepackage{avant}


%%----------------------------------------------------------------------------80
%% Settings
%%----------------------------------------------------------------------------80
% Hyperref package settings --------------------------------------------------80
\hypersetup{
  pdftitle={Programación con Fortran - Nivel Intermedio},
  pdfauthor={Mart\'{i}n Josemar\'{i}a Vuelta Rojas},
  pdfpagelayout=OneColumn,
  pdfnewwindow=true,
  pdfdisplaydoctitle=true,
  pdfstartview=XYZ,
  plainpages=false,
  hidelinks=false,
  unicode=true,
  bookmarksnumbered=true,
  bookmarksopen=true,
  bookmarksopenlevel=3,
  breaklinks=true,
  colorlinks=true,
  pdfborder={0 0 0}
}


% caption poackage settings --------------------------------------------------80
% \captionsetup[figure]{
%   labelfont=bf,
%   justification=centering
% }


% pgfplots package settings --------------------------------------------------80
\pgfplotsset{compat=1.15}
% \usepgfplotslibrary{dateplot}
% \usepgfplotslibrary{groupplots}


%%----------------------------------------------------------------------------80
%% Customizations
%%----------------------------------------------------------------------------80
% Cover shapes ---------------------------------------------------------------80
\backgroundsetup{
  scale=1,
  angle=0,
  opacity=1,
  contents={
  \begin{tikzpicture}[
      remember picture,
      overlay
    ]
    \path[
      fill=red-softbutterfly
    ] (current page.north west) rectangle (-8.5, 0);
    \path[
      fill=blue-softbutterfly
    ] (current page.south west) rectangle (-8.5, 0);
  \end{tikzpicture}
  }
}


% Fonts customization --------------------------------------------------------80
% Customizing chapter heading
% Font family: sans serif
\titleformat{\title}[display]{
  \normalfont\sffamily\Huge\bfseries\color{grey-800}
}{}{0pt}{}

% Customizing section heading
% Font family: sans serif
\titleformat{\section}{
  \normalfont\sffamily\large\bfseries\color{grey-800}
}{\thesection}{1em}{}

% Customizing section heading
% Font family: sans serif
\titleformat{\subsection}{
  \normalfont\sffamily\normalsize\bfseries\color{grey-800}
}{\thesubsection}{1em}{}


% Page geometry customization ------------------------------------------------80
% Customizing header spacings
\geometry{
  a4paper,
  headheight=25pt,
  headsep=12pt,
}


%%----------------------------------------------------------------------------80
%% Command definitions
%%----------------------------------------------------------------------------80
\makeatletter

% Colors ---------------------------------------------------------------------80
% Softbutterfly colors
\definecolor{red-softbutterfly}{HTML}{C83737}
\definecolor{blue-softbutterfly}{HTML}{162D50}

% Material design colors
\definecolor{white}{HTML}{FFFFFF}

\definecolor{red-500}{HTML}{F44336}
\definecolor{red-A700}{HTML}{D50000}

\definecolor{blue-50}{HTML}{E3F2FD}
\definecolor{blue-100}{HTML}{BBDEFB}
\definecolor{blue-800}{HTML}{1565C0}

\definecolor{grey-50}{HTML}{FAFAFA}
\definecolor{grey-800}{HTML}{424242}
\definecolor{grey-900}{HTML}{212121}

% Command for setting global color
\newcommand{\globalcolor}[1]{
  \color{#1}\global\let\default@color\current@color
}


% Fonts definitions ----------------------------------------------------------80
% Redefining default document font family
% Font family: serif
\renewcommand{\familydefault}{\rmdefault}


% Parragraph spacings --------------------------------------------------------80
% \setlength{\parindent}{4em}
% \setlength{\parskip}{0.5em}


% Table columns types --------------------------------------------------------80
\newcolumntype{L}[1]{>{\raggedright\let\newline\\\arraybackslash\hspace{0pt}}m{#1}}
\newcolumntype{C}[1]{>{\centering\let\newline\\\arraybackslash\hspace{0pt}}m{#1}}
\newcolumntype{R}[1]{>{\raggedleft\let\newline\\\arraybackslash\hspace{0pt}}m{#1}}

% Color box decoration -------------------------------------------------------80
\newcommand{\headerbox}[2][]{
  \tikz[overlay]
  \node[fill=blue!20,inner sep=4pt, anchor=text, rectangle, rounded corners=1mm, #1] {#2};
  \phantom{#2}
}

\newcommand{\scorebox}[2][]{
  \tikz[overlay]
  \node[fill=blue!20,inner sep=10pt, anchor=text, rectangle, rounded corners=1mm, #1] {#2};
  \phantom{#2}
}

% Clear double page definition -----------------------------------------------80
% Use of "This page is intentionally left in blank." for better UX
% Use of "Página intencionalmente dejada en blanco." for better UX
\def\cleardoubleintentionalpage{
  \clearpage%
  \if@twoside%
  \ifodd%
    \c@page%
  \else%
    \vspace*{\fill}%
    \begin{center}%
    % For spanish:
    \rm\emph{Página intencionalmente dejada en blanco.}
    % For english:
    % \rm\emph{This page is intentionally left in blank.}
    \end{center}%
    \vspace{\fill}%
    \thispagestyle{empty}%
    \newpage%
    \if@twocolumn%
    \hbox{}%
    \newpage%
    \fi%
  \fi%
  \fi%
}

% Completely white page
\def\cleardoublepage{
  \clearpage%
  \if@twoside%
  \ifodd%
    \c@page%
  \else%
    \vspace{\fill}%
    \thispagestyle{empty}%
    \newpage%
    \if@twocolumn%
    \hbox{}%
    \newpage%
    \fi%
  \fi%
  \fi%
}

% Flip margin ----------------------------------------------------------------80
\newcommand*{\flipmargins}{%
  \clearpage
  \setlength{\@tempdima}{\oddsidemargin}%
  \setlength{\oddsidemargin}{\evensidemargin}%
  \setlength{\evensidemargin}{\@tempdima}%
  \if@reversemargin
  \normalmarginpar
  \else
  \reversemarginpar
  \fi
}


% Page style definition ------------------------------------------------------80
% Plain Style
\fancypagestyle{plain}{%
  \fancyhead{}%
  \renewcommand{\headrulewidth}{0pt}
}

% Fancy Style
\fancypagestyle{fancy}{
  \fancyhf{}
  \renewcommand{\headrulewidth}{1pt}%
  \renewcommand{\footrulewidth}{0pt}%
  \fancyhead[EL]{
    % Make use of your own phrase
    Fortran
  }
  \fancyhead[OR]{
    % Make use of your own phrase
    \noindent\softbutterflylogo{3.5cm}\par
  }
  \fancyfoot[C]{
  {
  \begin{centering}
    \thepage
  \end{centering}
  }
  }
}

% noindent environment -------------------------------------------------------80
\newenvironment{noindented}{
  \par\setlength{\parindent}{0em}
}
{\par}

% Some useful symbols --------------------------------------------------------80
% Registered Mark
\newcommand{\registeredmark}{\textsuperscript{\small{\textregistered}}}

% SoftButterfly logo
\newcommand{\softbutterflylogo}[1]{
  \includegraphics[width=#1]{../resources/SoftButterfly-LaTeX-Logo.pdf}
}


\makeatother


%%----------------------------------------------------------------------------80
%% Document
%%----------------------------------------------------------------------------80
% Global document settings ---------------------------------------------------80
\AtBeginDocument{
  \globalcolor{grey-900}
}


% Document body --------------------------------------------------------------80
\begin{document}
  % BEGIN: Cover pager
  \begin{titlepage}
    \BgThispage
    \newgeometry{left=5cm,top=6cm,bottom=6cm,right=3.5cm}
    \begin{noindented}
      \setlength{\parskip}{0.5em}
      {
        \softbutterflylogo{7cm}
        \vspace{-1em}
      }
      \par
      {%
        \color{red-softbutterfly}
        \makebox[0pt][l]{
          \rule{1.3\textwidth}{1pt}
        }%
      }
      \par
      {
        \color{blue-softbutterfly}
        \textsf{
          \textbf{
            \textsc{Propuesta técnica de cruso de capacitación}
          }
        }
      }
      \par
      \vfill
      {
        \huge
        \textsf{Programación en Fortran}
      }
      \par
      {
        \large
        \textsf{Nivel intermedio}
      }
      \par
      \vfill
      {
        \color{blue-softbutterfly}
        \textsf{Sílabo elaborado por}
      }
      \par
      {
        \large
        \textsf{Martín Josemaría Vuelta Rojas}
      }
      \vskip\baselineskip
      \textsf{Lima, \today}
    \end{noindented}
    \afterpage{
      \cleardoubleintentionalpage
    }
  \end{titlepage}
  \restoregeometry

  % BEGIN: Report content
  \begin{noindented}
  {
    \huge\bfseries
    \textsf{%
      \textsc{Programación en Fortran}
    }
  }
  \par
  {
    \large
    \textsf{%
      \textsc{Nivel intermedio}
    }
  }
  \vspace{1em}
\end{noindented}

  %%----------------------------------------------------------------------------80
%% Section title
%%----------------------------------------------------------------------------80
\section{Introducción}


%%----------------------------------------------------------------------------80
%% Section Content
%%----------------------------------------------------------------------------80
Fortran es un lenguaje de programación desarrollado en los años 50 y activamente utilizado desde entonces en aplicaciones científicas y análisis numérico. Ha sido ampliamente adoptado por la comunidad científica para escribir aplicaciones con cómputos intensivos de alto rendimiento.

Desde 1958 ha pasado por varias versiones, entre las que destacan FORTRAN II, FORTRAN IV, FORTRAN 77, Fortran 90, Fortran 95, Fortran 2003 y Fortran 2008. Si bien el lenguaje era inicialmente un lenguaje imperativo, las últimas versiones incluyen elementos de la programación orientada a objetos.

El curso de Fortran (nivel básico), está orientado a que se aprendan los fundamentos de la programación en Fortran con énfasis en Fortran 90/95, por ser los estándares de mayor uso, y brindar un acercamiento a las nuevas técnicas de programación introducidas por los estándares más sin que ello afecte al rendimiento del las aplicaciones escritas con versiones anteriores.

  %%----------------------------------------------------------------------------80
%% Section title
%%----------------------------------------------------------------------------80
\section{Detalles}


%%----------------------------------------------------------------------------80
%% Section Content
%%----------------------------------------------------------------------------80
\begin{description}
  \item[Docente]\hfill\\
    Martín Josemaría Vuelta Rojas

    \begin{itemize}
      \item Linkedin: \url{www.linkedin.com/in/martinvuelta}
      \item GitHub: \url{www.github.com/zodiacfireworks}
      \item Correo electrónico: \href{mailto:martin.vuelta@gmail.com}{\texttt{martin.vuelta@gmail.com}}
    \end{itemize}

  \item[Nivel]\hfill\\
    Básico.

  \item[Metodología]\hfill
    \begin{itemize}
      \item Exposiciones dialogadas utilizando, apuntes, elementos de proyección fija.
      \item Talleres prácticos grupales en la elaboración de programas.
      \item Las clases serán eminentemente prácticas en una relación 70\% práctica - 30\% teórica.
    \end{itemize}

  \item[Duración]\hfill\\
    16 horas\footnote{Se sugiere una división en 4 sesiones de 4 horas con un receso de 10 minutos en cada sesión.}

  \item[Materiales a entregar]\hfill
    \begin{itemize}
      \item Manual de cada clase (PDF)
      \item Diapositivas de cada clase (PDF)
      \item \texttt{Cheatsheet} de Fortran (Impreso)
    \end{itemize}

  \item[Requerimientos]\hfill
    \begin{itemize}
      \item Proyector (con cable HDMI)
      \item Acceso a una conexión de Internet (WiFi o Cableado)
      \item Cada alumno deberá contar con su propio equipo para trabajar
    \end{itemize}
\end{description}

  %%----------------------------------------------------------------------------80
%% Section title
%%----------------------------------------------------------------------------80
\section{Temario}


%%----------------------------------------------------------------------------80
%% Section Content
%%----------------------------------------------------------------------------80
\begin{description}
  \item[Sesión 1. Introducción]\hfill
    \begin{enumerate}
      \item Fortran: Historia y estado actual
      \item Preparación del entorno de trabajo
      \item Conceptos básicos: \texttt{Hello world!} y el proceso de compilación
        \begin{itemize}
          \item \texttt{Hello world!}
          \item Compilación
          \item Enlazado
          \item Productos finales
        \end{itemize}
      \item Estructura de un programa en Fortran
      \item Formatos de escritura en Fortran
        \begin{itemize}
          \item Formato fijo
          \item Formato libre
        \end{itemize}
      \item Expresividad: El algoritmo TPK
    \end{enumerate}

  \item[Sesión 2. Instrucciones básicas]\hfill
    \begin{enumerate}
      \item Variables y tipos de datos
        \begin{itemize}
          \item Variables y Constantes
          \item Cadenas de caracteres
          \item Valores lógicos
          \item Enteros
          \item Reales (Precisión simple y doble)
          \item Complejo
          \item La clausula \texttt{IMPLICIT}
          \item Tipos derivados
        \end{itemize}

      \item Operaciones elementales
        \begin{itemize}
          \item Aritmética
          \item Comparación
          \item Lógica
          \item Funciones intrínsecas (\textit{builtin functions})
          \item Operaciones con caracteres
        \end{itemize}

      \item Estructuras de control
        \begin{itemize}
          \item Condiciones: \texttt{IF}
          \item Repeticiones
            \begin{itemize}
              \item \texttt{DO}
              \item \texttt{DO WHILE}
              \item \texttt{DO FROM X TO Y}
            \end{itemize}
          \item Selección: \texttt{SWITCH CASE}
          \item \textit{The infamous} \texttt{GO TO}
        \end{itemize}

      \item Instrucciones básicas de lectura y escritura de datos
        \begin{itemize}
          \item Escritura sobre la pantalla
          \item Entrada de datos por el usuario
          \item Entrada de datos por linea de comandos (Fortran 2003)
        \end{itemize}

      \item Proyecto 1: Proyecto Euler
    \end{enumerate}

  \item[Sesión 3. Arreglos / Lectura y escritura de archivos] \hfill
    \begin{enumerate}
      \item Declaración de arreglos
        \begin{itemize}
          \item Asignación de valores a arreglos
          \item Declaración dinámica de arreglos
        \end{itemize}
      \item Asignación en arreglos
        \begin{itemize}
          \item Subarreglos
          \item Expresiones de asignación y operaciones aritméticas
        \end{itemize}
      \item Instrucciones y operaciones exclusivas de arreglos
        \begin{itemize}
          \item Instrucciones de control
          \item Funciones intrínsecas
        \end{itemize}
      \item Operaciones matriciales

      \item Modificadores de formato
      \item Lectura y escritura de archivos
        \begin{itemize}
          \item Apertura de archivos
          \item Lectura y escritura en archivos
        \end{itemize}
      \item Ejecución de comandos del sistema operativo (Fortran 2008)

      \item Proyecto 2: Lectura de archivos de configuración
    \end{enumerate}

  \item[Sesión 4. Procedimientos / Módulos] \hfill
    \begin{enumerate}
      \item Subrutinas
        \begin{itemize}
          \item Argumentos ficticios
          \item Objetos locales
          \item Subrutinas internas
        \end{itemize}

      \item Funciones

      \item Utilidades
        \begin{itemize}
          \item Argumentos por nombre
          \item Argumentos opcionales
          \item Recursividad
          \item \texttt{PURE} \textit{keyword}
        \end{itemize}

      \item Módulos
        \begin{itemize}
          \item Datos y Objetos Compartidos
          \item Procedimientos de módulo
        \end{itemize}

      \item Interfaces genéricas
        \begin{itemize}
          \item Interfaces genéricas con procedimientos
          \item Interfaz operador
          \item Interfaz de asignación
        \end{itemize}

      \item Proyecto 3: Manipulación de fechas y series de tiempo
    \end{enumerate}

  \item[Sesión 5: Tópicos adicionales] \hfill
    \begin{enumerate}
      \item Punteros
      \item Programación orientada a objetos
      \item Tópicos adicionales
      \begin{itemize}
        \item Opciones de compilador
        \item \textit{Makefiles}
        \item Control de versiones
      \end{itemize}
      \item Proyecto 4: Derivación e integración numérica
    \end{enumerate}
\end{description}

\end{document}
