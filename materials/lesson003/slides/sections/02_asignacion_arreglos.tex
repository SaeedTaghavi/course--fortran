%-----------------------------------------------------------------------------80
% SECTION TITLE|
%-----------------------------------------------------------------------------80

\section{Asignación en arreglos}  

%-----------------------------------------------------------------------------80
% CONTENT
%-----------------------------------------------------------------------------80

\subsection{Subarreglos-Expresiones y operaciones}


\begin{frame}[fragile]{Asignación en arreglos}
 \begin{itemize}[<+(0)->]
  \item La asignación de valores a un arreglo también puede darse por subarreglos, es decir, bloques de arreglos o arreglos locales.
  \vspace{0.15 cm}
  \item [] 
  \begin{minted}[linenos,autogobble]{fortran}
  REAL, DIMENSION(3,2)::matriz1
  REAL, DIMENSION(-1:1,0:1)::matriz2
  :
  matriz1(1,1)=1        ! asignacion a un solo elemento
  matriz1=1             ! asignacion a todo el tablero con el valor 1
  matriz2=matriz(2,2)   ! asignacion de todo el tablero matriz2 con
                        ! el valor de matriz(2,2)
  matriz2=matriz1       ! asignacion, copiando los valores de la matriz
  \end{minted}
 \end{itemize}
\end{frame}

%Subarreglos------------------------------------------------------------------80

\begin{frame}[fragile]{Asignación en arreglos}
 \textbf{Subarreglos}
  \begin{itemize}[<+(1)->]
  \item Los subarreglos se identifican mediante el uso de la sintaxis \\ 
   \centering <inicio>:<final>:<incremento>
  \item [-] Si <inicio> se omite (:), toma el primer valor del arreglo. 
  \item [-] Si <final> se omite (:), toma el último valor del arreglo.
  \item [-] Si <incremento> se omite (:), toma el valor de 1.
  \vspace{0.15cm}
  \item[]
  \begin{minted}[linenos,autogobble]{fortran}
    REAL, DIMENSION(1:3)::vector
    REAL, DIMENSION(1:3,1:3)::matriz
    :
    WRITE(*,*)vector(1:2)        !los dos primeros elementos
    WRITE(*,*)matriz(1:1,1:3)    !la primera fila
    WRITE(*,*)matriz(1:3:2,1:3)  !la primera y tercera fila
    WRITE(*,*)matriz(3:1:-2,1:3) !la tercera y primera fila
    WRITE(*,*)matriz(:2,:2)      !el primer bloque 2x2
    :
  \end{minted}
 \end{itemize}
\end{frame}

%Expresiones y operaciones----------------------------------------------------80

\begin{frame}[fragile]{Asignación en arreglos}
 \textbf{Expresiones de asignación y operaciones aritméticas}
  \begin{itemize}[<+(1)->]
  \item Las expresiones entre arreglos  deben crear arreglos de la misma forma.
  \item Los elementos de los arreglos están relacionados por relaciones de orden, prestablecida sobre los índices, como se evidencia en una matriz por las columnas y filas.
  \item Las operaciones aritméticas y funciones intrínsecas ($\cos, \exp$, etc.) operan por elemento.
  \vspace{0.15 cm} 
  \item []
   \begin{minted}[linenos,autogobble]{fortran}
    REAL, DIMENSION(1:3)::vect          !vector
    REAL, DIMENSION(1:3,1:3)::matr1     !matriz 1
    REAL, DIMENSION(0:2,-1:1)::matr2    !matriz 2
    :
    matr1(:,1)=matr2(0,:)+vect   !El resultado va a la columna1 la de matr1, de 
                                 !la suma de la fila0 de la matr2 con el vector.
    matr1=matr1*matr2            !Los productos son calculados individualmente
    matr1=cos(matr2)
    matr1=exp(matr2)
    matr2(:,0)=sqrt(vector)
   \end{minted}
 \end{itemize}
\end{frame}


\begin{frame}[fragile]{Asignación en arreglos}
 \textbf{Vectores sub-índices}
  \begin{itemize}[<+(1)->]
  \item Es posible también identificar los elementos de un arreglo a través de un vector índice, de elementos tipos INTEGER. 
  \vspace{0.15cm}
  \item []
   \begin{minted}[linenos,autogobble]{fortran}
    INTEGER, DIMENSION(1:3)::indice::(/2,4,6/)
    REAL, DIMENSION(1:10,1:10)::matriz
    :
    PRINT*,matriz(5,indice)       !escribe los elementos (5,2), (5,4) y (5,6)
    PRINT*,matriz(indice,indice)  !los elementos en el orden (2,2), (4,2), (6,2)
                                  !(2,4), (4,4), (6,4), (2,6), (4,6) y (6,6)
    matriz(indice,5)=matriz(1:5:2,6)    !Se asigna a matriz (2,5) el
                                        !valor de matriz(1,6),
                                        !a  matriz(4,5) el valor matriz(3,6)
                                        !y a matriz(6,5) el valor matriz(5,6)
   \end{minted}
 \end{itemize}
\end{frame}