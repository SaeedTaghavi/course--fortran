%-----------------------------------------------------------------------------80
% SECTION TITLE|
%-----------------------------------------------------------------------------80

\section{Instrucciones y operaciones exclusivas de arreglos}  

%-----------------------------------------------------------------------------80
% CONTENT
%-----------------------------------------------------------------------------80

\subsection{Instrucciones de control / Funciones intrínsecas}


\begin{frame}[fragile]{Instrucciones de control} 
    \begin{itemize}[<+(0)->]
        \item Fortran permite asignar de valores en elementos específicos de un arreglo, bajo una determinada condición, con la instrucción WHERE. 
        \item La sintaxis es la siguiente \\
            \centering WHERE (<arreglo control>) <arreglo>=<expresión> 
        \item [] donde los elementos de <arreglo control> y <arreglo> son tipo LOGICAL.
        \item [] Por ejemplo: Sea A un arreglo de $2 \times 2$ 
            $$
                A = \left( \begin{array}{cc}
                    100. & 10. \\
                    1. & 0. \end{array} \right)
            $$
        \item [] 
            \begin{minted}[linenos,autogobble]{fortran}
                :
                REAL, DIMENSION(2,2)::A,B
                :
                WHERE(A>0) B=log10(A)
                : 
            \end{minted}
        \item [] dará como resultado
            $$
                A = \left( \begin{array}{cc}
                    2. & 1. \\
                    0. & 0. \end{array} \right)
            $$
    \end{itemize}
\end{frame}


\begin{frame}[fragile]{Instrucciones de control} 
    \begin{itemize}[<+(0)->]
        \item Es posible permitir asignaciones a valores de tipo LOGICAL, del arreglo de control, utilizando la instrucción ELSE WHERE.
        \item La sintaxis es la siguiente: \\
          \vspace{0.1cm}
            \begin{minted}[linenos,autogobble]{fortran}
            WHERE (<arreglo control>)
                   <bloque de instrucciones> 
            ELSE WHERE 
                   <bloque de instrucciones>
            END WHERE
            \end{minted}
        \item [] Del ejemplo anterior, tenemos:
            \begin{minted}[linenos,autogobble]{fortran}
                :
                REAL, DIMENSION(2,2)::A,B
                :
                B=A
                :
                WHERE(A>0) 
                    B=log10(A)
                ELSE WHERE
                    B=-100
                ELSE WHERE
                :
            \end{minted} 
    \end{itemize}
\end{frame}


\begin{frame}[fragile]{Instrucciones de control} 
    \begin{itemize}[<+(0)->]
        \item [] que dará como resultado
            $$
                B = \left( \begin{array}{cc}
                    2. & 1. \\
                    0. & -100. \end{array} \right)
            $$
        \item Considerando un arreglo de control \emph{Acontrol} de tipo LOGICAL, tenemos:
    \end{itemize}
    \onslide<3->\textbf{Algunas funciones de asignación de control}
    \begin{itemize}[<+(1)->]
        \item all \\ 
        \begin{minted}[linenos,autogobble]{fortran}
        all(Acontrol)       !valor .true. si todos los elementos de Acontrol
                            !tienen valor .true.. Sino el valor es .false.
        \end{minted}
        \item any \\ 
        \begin{minted}[linenos,autogobble]{fortran}
        any(Acontrol)       !valor .true. si al menos un elemento de Acontrol
                            !tiene valor .true.. Sino el valor es .false.
        \end{minted}
        \item count \\ 
        \begin{minted}[linenos,autogobble]{fortran}
        count(Acontrol)     !valor INTEGER indicando el número de elementos 
                            !de Acontrol, cuyos valores son .true.
        \end{minted}                         
    \end{itemize}
\end{frame}


\begin{frame}[fragile]{Instrucciones de control}
    \begin{itemize}[<+(0)->]
        \item Agregando la opción \emph{dim} a las funciones de control, es posible reducir los arreglos a unos de forma unidimensional (vectores).
        \item [] Por ejemplo: 
            $$
                Acontrol = \left( \begin{array}{cc}
                    .true.  & .false.  \\
                    .false. & .false. \\
                    .true.  & .true. \end{array} \right)
            $$
        \item Considerando un arreglo de control \emph{Acontrol} de tipo LOGICAL, tenemos:
         \vspace{0.12cm}
            \begin{minted}[linenos,autogobble]{fortran}
                all(Acontrol,dim=1)     !da (/.false.,.false.,.true./)
                all(Acontrol,dim=2)     !da (/.false.,.false./)
                any(Acontrol,dim=1)     !da (/.true.,.false.,.true./)
                any(Acontrol,dim=2)     !da (/.true.,.true./)
                count(Acontrol,dim=1)   !da (/1,0,2/)
                count(Acontrol,dim=2)   !da (/2,1/)
            \end{minted}                         
    \end{itemize}
\end{frame}


\begin{frame}[fragile]{Funciones intrínsecas} 
    \begin{itemize}[<+(0)->]
        \item Fortran, en sus versiones 90 y posteriores, cuenta con funciones especiales para arreglos.
        \item [] Tomando como ejemplo el siguiente arreglo:
            $$
                A = \left( \begin{array}{ccc}
                    5.      &  3.   &  1.  \\
                    8.      & 12.   & 10.  \\
                    9.      & 11.   &  7.  \\
                    4.      &  6.   &  2.  \\
                \end{array} \right)
            $$
            cuya declaración es: \\ 
            \vspace{0.1cm}
            \begin{minted}[linenos,autogobble]{fortran}
                REAL, DIMENSION(0:3,2:4)::A
            \end{minted}
        \item [] \textbf{lbound} 
        \item Da como resultado un vector de tipo INTEGER, cuyos valores son los mínimos que pueden tomar los índices del arreglo.
             \vspace{0.1cm}
            \begin{minted}[linenos,autogobble]{fortran}
                lbound(A) !asigna los valores (/0,2/)
            \end{minted}
        \item [] Agregando la opción \emph{dim} se obtiene el mínimo valor de cada índice.
            \vspace{0.1cm}
            \begin{minted}[linenos,autogobble]{fortran}
                lbound(A,dim=1) !se obtiene: 0
                lbound(A,dim=2) !se obtiene: 2
            \end{minted}
    \end{itemize}
\end{frame}


\begin{frame}[fragile]{Funciones intrínsecas} 
    \begin{itemize}[<+(0)->]
        \item [] \textbf{ubound} 
        \item Da como resultado un vector de tipo INTEGER, cuyos valores son los máximos que pueden tomar los índices del arreglo.
             \vspace{0.1cm}
            \begin{minted}[linenos,autogobble]{fortran}
                ubound(A) !asigna los valores (/3,4/)
            \end{minted}
        \item [] Agregando la opción \emph{dim} se obtiene el máximo valor de cada índice.
            \vspace{0.1cm}
            \begin{minted}[linenos,autogobble]{fortran}
                ubound(A,dim=1) !se obtiene: 3
                ubound(A,dim=2) !se obtiene: 4
            \end{minted}
        \item [] \textbf{size} 
        \item Proporciona el número de elementos de un arreglo valor tipo INTEGER.
            \vspace{0.1cm}
            \begin{minted}[linenos,autogobble]{fortran}
                size(A) !se obtiene: 12
            \end{minted}
        \item [] Agregando la opción \emph{dim} se obtiene la longitud del rango del índice indicado.
             \vspace{0.1cm}
            \begin{minted}[linenos,autogobble]{fortran}
                size(A,dim=1) !se obtiene: 4
                size(A,dim=2) !se obtiene: 3
            \end{minted}
    \end{itemize}
\end{frame}


\begin{frame}[fragile]{Funciones intrínsecas} 
    \begin{itemize}[<+(0)->]
        \item [] \textbf{shape} 
        \item Proporciona la forma del arreglo en un vector de tipo INTEGER cuyo tamaño es el número de índices y donde cada elemento del vector representa la longitud del rango del índice.
             \vspace{0.1cm}
            \begin{minted}[linenos,autogobble]{fortran}
                shape(A) !se obtiene: (/4,3/)
            \end{minted}
        \item [] \textbf{minval, maxval} 
        \item Proporciona el mínimo o máximo valor, respectivamente, de un arreglo de tipo numérico.
            \vspace{0.1cm}
            \begin{minted}[linenos,autogobble]{fortran}
                minval(A) !se obtiene: 12
                manval(A) !se obtiene: 1
            \end{minted}
        \item [] Agregando la opción \emph{dim} se obtiene un vector cuyos valores son los mínimos o máximos fijados por el valor del índice.
            \vspace{0.1cm}
            \begin{minted}[linenos,autogobble]{fortran}
                minval(A, dim=1) !se obtiene: (/1.,8.,7.,2./)
                manval(A, dim=1) !se obtiene: (/5.,12.,11.,6./)
                minval(A, dim=2) !se obtiene: (/4.,3.,1./)
                manval(A, dim=2) !se obtiene: (/9.,12.,10./)
            \end{minted}
    \end{itemize}
\end{frame}


\begin{frame}[fragile]{Funciones intrínsecas} 
    \begin{itemize}[<+(0)->]
        \item [] \textbf{minloc, maxloc}
        \item Proporcionan los índices del valor mínimo y máximo, respectivamente, de los elementos del arreglo.
            \vspace{0.1cm}
            \begin{minted}[linenos,autogobble]{fortran}
                minloc(A) !se obtiene: (/1,3/)
                maxloc(A) !se obtiene: (/2,2/)
            \end{minted}
        \item [] \textbf{sum, product}
        \item La función sum nos da la suma de todos los elementos del arreglo, mientras la opción product, el producto de los mismos, siempre y cuando el arreglo sea de tipo numérico.
            \vspace{0.1cm}
            \begin{minted}[linenos,autogobble]{fortran}
                sum(A)      !se obtiene: 4.7900160E+08
                product(A)  !se obtiene: 78.00000
            \end{minted}
        \item [] Agregando la opción \emph{dim}, se restringen las operaciones de suma y multiplicación a los índices.
            \begin{minted}[linenos,autogobble]{fortran}
                sum(A, dim=1)       !se obtiene: (/26.0000,32.0000,20.0000)
                sum(A, dim=2)       !se obtiene: (/9.000000,30.0000,27.00000,12.00000)
                product(A, dim=1)   !se obtiene: (/1440.000,2376.000,140.0000)
                product(A, dim=2)   !se obtiene: (/15.00000,960.00000,693.0000,48.00000)
            \end{minted}
    \end{itemize}
\end{frame}



