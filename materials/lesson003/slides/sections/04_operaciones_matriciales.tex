%-----------------------------------------------------------------------------80
% SECTION TITLE|
%-----------------------------------------------------------------------------80

\section{Operadores matriciales}  

%-----------------------------------------------------------------------------80
% CONTENT
%-----------------------------------------------------------------------------80

\subsection{Operaciones matriciales}

\begin{frame}[fragile]{Operadores matriciales} 
    \begin{itemize}[<+(0)->]
        \item Fortran, en sus versiones 90 y posteriores, permite realizar operaciones con matrices y vectores (arreglos bidimesionales y unidimensionales, respectivamente) como la adición, sustracción y multiplicación por escalar.
        \item [] \textbf{matmul}
        \item La función \emph{matmul} permite efectuar la multiplicación entre estos arreglos.
        \item La sintaxis es la siguiente: \\
        \vspace{0.1cm}
            \begin{minted}[linenos,autogobble]{fortran}
                matmul(<matriz>,<matriz>)   !se obtiene una matriz
                matmul(<matriz>,<vector>)   !se obtiene un vector
            \end{minted}
        \item [] Por ejemplo: \\
        Sean los arreglos
            \begin{equation*}
            A = \begin{pmatrix}
                    1. & 2. & 3.\\
                    4. & 5. & 6.
                \end{pmatrix},  
            B = \begin{pmatrix}
                    1. & 2.\\
                    3. & 4.\\
                    5. & 6.
                \end{pmatrix},  
            u = \begin{pmatrix}
                    1.\\
                    2.\\
                    3.
                \end{pmatrix},  
            v = \begin{pmatrix}
                    1.\\
                    2.  
                \end{pmatrix}.
            \end{equation*} 
    \end{itemize}
\end{frame}


\begin{frame}[fragile]{Operadores matriciales} 
    \begin{itemize}[<+(0)->]
        \item [] El siguiente ejemplo muestra el uso de \emph{matmul}:
        \vspace{0.1cm}
            \begin{minted}[linenos,autogobble]{fortran}
                PROGRAM matmul
                REAL, DIMENSION(3)::u=(/1.,2.,3./),x
                REAL, DIMENSION(2)::v=(/1.,2./),y
                REAL, DIMENSION(3,2)::A=reshape((/(1.*i,i=1,6)/),(/3,2/))
                REAL, DIMENSION(2,3)::B=reshape((/(1.*i,i=1,6)/),(/2,3/))
                REAL, DIMENSION(3,3)::C
                REAL, DIMENSION(2,2)::D
                C=matmul(A,B)
                D=matmul(B,A)
                y=matmul(B,u)
                x=matmul(A,v)
                :
            \end{minted}
        \item [] dando como resultado
        \small
            \begin{equation*}
            C = \begin{pmatrix}
                    9,000000 & 19,00000 & 29,00000\\
                    12,00000 & 26,00000 & 40,00000\\
                    15,00000 & 33,00000 & 51,00000    
                \end{pmatrix},    
            D = \begin{pmatrix}
                    22,00000 & 49,00000\\
                    28,00000 & 64,00000
                \end{pmatrix}  
            \end{equation*}
            \begin{equation*}
            x = \begin{pmatrix}
                    9,000000\\
                    12,00000\\ 
                    15,00000
                \end{pmatrix},  
            y = \begin{pmatrix}
                    22,00000\\
                    28,00000
                \end{pmatrix}
            \end{equation*}      
    \end{itemize}
\end{frame}


\begin{frame}[fragile]{Operadores matriciales} 
    \begin{itemize}[<+(0)->]
        \item [] \textbf{dot\_product}
        \item El producto escalar de dos vectores del mismo tamaño se realiza empleando la función \emph{dot\_product}:.
        \item La sintaxis es la siguiente: \\
            \vspace{0.1cm}
            \begin{minted}[linenos,autogobble]{fortran}
                dot_product(<vector>,<vector>)
            \end{minted}
        \item [] Por ejemplo:
            \vspace{0.1cm}
            \begin{minted}[linenos,autogobble]{fortran}
                :
                REAL::valor
                REAL, DIMENSION(8)::vector=(/(1.*i,i=1,8)/)
                :
                valor=dot_product(vector(1:7:2),vector(2:8:2)) !se obtiene: 100
                :
            \end{minted}    
        \item [] \textbf{transpose}
        \item La transposición de una matriz, el intercambio de posición simétrica entre filas y columnas, es posible mediante la función \emph{transpose}.
        \item La sintaxis es la siguiente: \\
            \vspace{0.1cm}
            \begin{minted}[linenos,autogobble]{fortran}
                transpose(<matriz>)
            \end{minted}
    \end{itemize}
\end{frame}



