%-----------------------------------------------------------------------------80
% SECTION TITLE
%-----------------------------------------------------------------------------80
\section{Preliminares}


%-----------------------------------------------------------------------------80
% CONTENT
%-----------------------------------------------------------------------------80
% Acerca del curso -----------------------------------------------------------80
\subsection{Acerca del curso}
\begin{frame}[fragile]{Acerca del curso}
  \textbf{El curso básico de Fortran tiene por objetivo}
  \begin{itemize}[<+(1)->]
    \item Que obtengan un conocimiento sólido de las carácteristicas de Fortran.
    \item Que se familiarizen con el flujo de infromación en los programas desarrolados en Fortran.
    \item Que tengan una introducción nuevas caracteristicas de los estándares recientes Fortran 2003 y 2008.
    \item Que puedan contruir su entorno de darrollo para programar en Fortran cómodamente.
  \end{itemize}
\end{frame}


\begin{frame}[fragile]{Acerca del curso}
  \textbf{Metodología}
  \begin{itemize}[<+(1)->]
    \item Exposiciones dialogadas utilizando, apuntes, elementos de proyección fija.
    \item Talleres prácticos grupales en la elaboración de programas.
    \item Las clases serán eminentemente prácticas en una relación 70\% práctica - 30\% teórica.
  \end{itemize}
\end{frame}

\begin{frame}[fragile]{Acerca del curso}
  \textbf{Referencias}
  \begin{itemize}[<+(1)->]
    \item I. Chivers, J. Sleightholme, \textit{Introduction to Programming with Fortran. With Coverage of Fortran 90, 95, 2003, 2008 and 77}, Springer-Verlag London, 2012.
    \item Michael Metcalf, John Reid, Malcolm Cohen, \textit{Modern Fortran Explained}, Oxford University Press, USA 2011
    \item Morten Hjorth-Jensen, \textit{Computational Physics, Lecture Notes Fall 2015}, Department of Physics, University of Oslo. 2015.
  \end{itemize}
\end{frame}

\begin{frame}[fragile]{Acerca del curso}
  \textbf{Materiales}
  \begin{itemize}[<+(1)->]
    \item[] Repositorio del curso
    \begin{itemize}
      \item[\faGithub] \href{https://github.com/zodiacfireworks/course--fortran-basic}{https://github.com/zodiacfireworks/course--fortran-basic}
      \item[\faGithub] \href{https://github.com/zodiacfireworks/course--fortran-intermediate}{https://github.com/zodiacfireworks/course--fortran-intermediate}
    \end{itemize}
  \end{itemize}

  \onslide<5->\begin{alertblock}{¡Importante!}
    \begin{itemize}[<+(2)->]
      \item[] Cada alumno debera tener una cuenta en GitHub
      \begin{itemize}
        \item[\color{black}\faGithub] \href{https://github.com}{https://github.com}
      \end{itemize}
    \end{itemize}
  \end{alertblock}
\end{frame}


% Acerca del instructor ------------------------------------------------------80
\subsection{Acerca del instructor}
\begin{frame}[fragile]{Acerca del instructor}
  \begin{center}
    \textbf{Martín Josemaría Vuelta Rojas}
  \end{center}

  \begin{itemize}[<+(1)->]
    \item Software Developer
    \item Web Developer
    \item Investigador
    \item Programador
    \begin{itemize}
      \item En investigación:
        \begin{itemize}
          \item C, C++, Fortran, Python, R, Julia, Mathematica, Matlab, LaTeX
        \end{itemize}
      \item En web:
        \begin{itemize}
          \item HTML, CSS, JavaScript, Python
        \end{itemize}
      \item En mobile:
        \begin{itemize}
          \item Kotlin, Java, C++
        \end{itemize}
      \item Hobbie:
        \begin{itemize}
          \item Scala, Pixie, Clojure, Elixir, Haskel, Oz, Kotlin, ...
        \end{itemize}
    \end{itemize}
  \end{itemize}
\end{frame}

\begin{frame}[fragile]{Acerca del instructor}
  \begin{center}
    \textbf{Martín Josemaría Vuelta Rojas}
  \end{center}

  \begin{itemize}[<+(1)->]
    \item SoftButterfly
    \item HackSpace Perú
    \item Jupyter Notebook
    \item Fedora
    \item GNOME
    \item UNMSM
  \end{itemize}
\end{frame}

\begin{frame}[fragile]{Acerca del instructor}
  \begin{center}
    \textbf{Martín Josemaría Vuelta Rojas}
  \end{center}

  \begin{itemize}[<+(1)->]
    \item[\faMobile]   \begin{center}
      \href{tel:+51982042088}{+51 982 042 088}
    \end{center}
    \item[\faEnvelope] \begin{center}
      \href{mailto:martin.vuelta@gmail.com}{martin.vuelta@gmail.com}
    \end{center}
    \item[\faLinkedin] \begin{center}
      \href{https://www.linkedin.com/in/martinvuelta/}{martinvuelta}
    \end{center}
    \item[\faGithub]   \begin{center}
      \href{https://github.com/zodiacfireworks}{zodiacfireworks}
    \end{center}
  \end{itemize}
\end{frame}
