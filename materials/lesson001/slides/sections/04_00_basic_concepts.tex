%-----------------------------------------------------------------------------80
% SECTION TITLE
%-----------------------------------------------------------------------------80
\section{Conceptos básicos}


%-----------------------------------------------------------------------------80
% CONTENT
%-----------------------------------------------------------------------------80
%Definición-------------------------------------------------------------------80


\subsection{Hello world!}
\begin{frame}[fragile]{Hello world! y el proceso de compilación}
 \textbf{Hello world!}
  \begin{itemize}[<+(1)->]
   \item Se llama programa a un conjunto de instrucciones, realizadas computacionalmente en un tiempo determinado, aplicadas en la introducción, procesamiento o salida de datos.   
   \item Un programa en fortran tiene la siguiente forma:
   \vspace{6pt}
   \item []
    \begin{minted}[linenos,autogobble]{fortran}
     PROGRAM hello_world
        WRITE(*, *) Message
     END PROGRAM hello_world
    \end{minted}
  \end{itemize}
\end{frame}

\begin{frame}[fragile]{Hello world! y el proceso de compilación}
 \textbf{Scientific Hello world!}
  \begin{itemize}[<+(1)->]
   \vspace{6pt}
   \item []
    \begin{minted}[linenos,autogobble]{fortran}
    PROGRAM hello_world
    IMPLICIT NONE

    ! Angulo de entrada
    REAL(KIND=4) :: theta
    
    ! Resultado de aplicar la función seno 
    REAL(KIND=4) :: sin_of_theta
    
    ! Mensaje
    CHARACTER(len=*), PARAMETER :: Message = 'Hello World'

    WRITE(*, *) 'Ingrese un ángulo [rad]: '
    READ(*, *) theta
    sin_of_theta = SIN(theta)
    
    WRITE(*, *) Message
    WRITE(*, *) "sin(", theta, ") = ", sin_of_theta
    END PROGRAM hello_world
    \end{minted}
  \end{itemize}
\end{frame}

